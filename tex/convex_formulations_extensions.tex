\documentclass{article}
\usepackage[utf8]{inputenc}
\usepackage{amsmath}
\usepackage{amsthm}
\usepackage{amssymb}
\usepackage{xcolor}
\newtheorem{theorem}{Theorem}
\newtheorem{proposition}{Proposition}
\newtheorem{definition}{Definition}
\newtheorem{corollary}{Corollary}
\newtheorem{lemma}{Lemma}

\title{Properties of Chance-Constrained Stochastic Electricity Markets}
\author{Mathias Berger}
\date{Spring 2022}

\begin{document}

\maketitle

\section{Introduction}

We analyse the properties of a chance-constrained stochastic market for energy and reserves involving stochastic wind power producers, flexible generators and inflexible electricity consumers. 

\section{Problem Statement}

\subsection{Preliminaries}

We consider four types of agents, namely flexible electricity generators, stochastic wind power producers, inflexible electricity consumers and a market operator, which we describe further below.

\textit{Wind Producers}: We consider $N \in \mathbb{N}$ stochastic wind producers. Although a forecast $\tilde{W}_i \in \mathbb{R}_+$ is available for the production of wind farm $i = 1, \ldots, N$, the actual output $\mathbf{p}_i^w(\boldsymbol{\omega}_i) \in \mathbb{R}_+$ may deviate from $\tilde{W}_i$ by some amount given by random variable $\boldsymbol{\omega}_i \in \Omega_i \subseteq \mathbb{R}$, such that $\mathbf{p}_i^w(\boldsymbol{\omega}_i) = \tilde{W}_i + \boldsymbol{\omega}_i$. Note that we use bold symbols to denote random variables. The first and second-order moments (i.e., the mean and variance) of the distribution of the forecast error $\boldsymbol{\omega}_i$ are denoted by $\mathbb{E}[\boldsymbol{\omega}_i] = \mu_i$ and $\mbox{Var}[\boldsymbol{\omega}_i] = \sigma_i^2$, respectively. The covariance between $\boldsymbol{\omega}_i$ and $\boldsymbol{\omega}_j$ is denoted by $\mbox{Cov}[\boldsymbol{\omega}_i, \boldsymbol{\omega}_j] = \kappa_{ij} = \rho_{ij} \sigma_i \sigma_j$, with $\rho_{ij}$ the Pearson correlation coefficient. All agents are assumed to have access to the same amount of information about $\{\boldsymbol{\omega}_i\}_{\forall i}$ and $\{\mathbf{p}_i^w(\boldsymbol{\omega}_i)\}_{\forall i}$. Wind producers are assumed to have zero marginal cost, be paid for their forecast production and charged for the provision of reserves.

\textit{Flexible Generators}: We consider $K \in \mathbb{N}$ flexible generators. Each generator may produce electricity and contribute to the provision of reserves in the system. Hence, the actual power output $\mathbf{p}_k^g(\{\boldsymbol{\omega}_i\}_{\forall i})$ is given by an affine control law. For generator $k$, this law can be expressed as $\mathbf{p}_k^g(\{\boldsymbol{\omega}_i\}_{\forall i}) = p_k - \sum_{i }\alpha_{ki} \boldsymbol{\omega}_i$, where $p_k$ denotes the scheduled power output and $\alpha_{ki}$ is the share of the forecast error of wind producer $i$ covered by generator $k$ through the reserve mechanism. It is assumed that generators are paid for their scheduled production $p_k$ (not their real-time production $\mathbf{p}_k^g(\{\boldsymbol{\omega}_i\}_{\forall i})$) in the energy market and their contributions $\{\alpha_{ki}\}_{\forall i}$ to reserve procurement. Since the actual power output of generators directly depends on the uncertain wind production and only becomes known when $\{\boldsymbol{\omega}_i\}_{\forall i}$ are revealed, it must be ensured that power generation bounds are not exceeded. Chance (i.e., probabilistic) constraints can be used for this purpose, and $\epsilon_k$ denotes the tolerance for constraint violations (e.g., constraints may be violated $100 \times \epsilon_k \%$ of the time). Generator $k$ is assumed to have linear and quadratic marginal production cost components denoted by $c_k^L \in \mathbb{R}_+$ and $c_k^Q \in \mathbb{R}_+$, respectively.

\textit{Inflexible Consumers}: We consider a set of consumers with aggregate demand $D \in \mathbb{R}_+$, which is assumed inelastic and known with certainty.

\textit{Market Operator}: In an equilibrium problem, a market operator (also known as a \textit{social planner}) can be interpreted as an auctioneer seeking to identify prices for energy and reserves so that the markets for energy and reserves clear (akin to a Walras auctioneer \cite{Uzawa1960}). 

\section{Equilibrium Problem Formulation}

The stochastic electricity market can be modelled as a stochastic equilibrium problem where flexible generators seek to maximise their utility while being coupled through market clearing constraints. The different components of the equilibrium problem formulation are described in this section. 

\subsection{Flexible Generators}

We start by describing the stochastic optimisation problems faced by flexible generators, and recast them as deterministic optimisation problems.

\subsubsection{Stochastic Formulation}

We consider risk-neutral flexible generators (i.e., generators only care about the expected pay-off and not its variance). Thus, flexible generator $k$ seeks to maximise its expected profit subject to chance constraints on its power output:
\begin{align}
\underset{p_k, \{\alpha_{ki}\}_{\forall i}}{\max} \hspace{10pt} & \mathbb{E}[\lambda p_k + \sum_i \chi_i \alpha_{ki} - c_k^L \mathbf{p}_k^g(\{\boldsymbol{\omega}_i\}_{\forall i}) - c_k^Q (\mathbf{p}_k^g(\{\boldsymbol{\omega}_i\}_{\forall i}))^2]\\
\mbox{s.t. } & \mathbb{P}[\mathbf{p}_k^g(\{\boldsymbol{\omega}_i\}_{\forall i}) \le \overline{p}_k] \ge 1 - \epsilon_k,\\
& \mathbb{P}[0 \le \mathbf{p}_k^g(\{\boldsymbol{\omega}_i\}_{\forall i})] \ge 1 - \epsilon_k,\\
&p_k \in \mathbb{R}, 0 \le \alpha_{ki} \le 1, \forall i.
\end{align}
Note that electricity and reserve prices, which are denoted by $\lambda$ and $\{\chi_i\}_{\forall i}$, respectively, are treated as parameters by each producer $k$.
\subsubsection{Deterministic Equivalent Formulation}

The stochastic expected profit maximisation problem faced by producer $k$ can be recast as an equivalent deterministic program as follows. First, by linearity of the expectation operator, the expected profit can be re-written as
\begin{align*}
\Pi_k(p_k, \{\alpha_{ki}\}_{\forall i}) &= \mathbb{E}\Big[\lambda p_k + \sum_i \chi_i \alpha_{ki} - c_k^L \mathbf{p}_k^g(\{\boldsymbol{\omega}_i\}_{\forall i}) - c_k^Q (\mathbf{p}_k^g(\{\boldsymbol{\omega}_i\}_{\forall i}))^2\Big]\\
&= \mathbb{E}\big[\lambda p_k\big] + \mathbb{E}\Big[\sum_i \chi_i \alpha_{ki}\Big] - \mathbb{E}\big[c_k^L \mathbf{p}_k^g(\{\boldsymbol{\omega}_i\}_{\forall i})\big] - \mathbb{E}\big[c_k^Q (\mathbf{p}_k^g(\{\boldsymbol{\omega}_i\}_{\forall i}))^2\big].
\end{align*}
We focus on each term separately and obtain
\begin{align*}
\mathbb{E}[\lambda p_k] =& \lambda p_k,\\
\mathbb{E}\Big[\sum_i \chi_i \alpha_{ki}\Big] =& \sum_i \chi_i \alpha_{ki},\\
\mathbb{E}[c_k^L \mathbf{p}_k^g(\{\boldsymbol{\omega}_i\}_{\forall i})] &= \mathbb{E}\Big[c_k (p_k - \sum_i \alpha_{ki} \boldsymbol{\omega}_i)\Big]\\
=& c_k^L\Big(p_k - \sum_i \alpha_{ki} \mu_i\Big),\\
\mathbb{E}[c_k^Q (\mathbf{p}_k^g(\{\boldsymbol{\omega}_i\}_{\forall i}))^2] &= \mathbb{E}[c_k^Q (p_k - \sum_i \alpha_{ki} \boldsymbol{\omega}_i)^2]\\
=& \mathbb{E}\Big[c_k^Q \big(p_k^2 - \sum_i 2 \boldsymbol{\omega}_i \alpha_{ki} p_k + \sum_i \sum_{j \ne i} \alpha_{ki} \alpha_{kj}  \boldsymbol{\omega}_i  \boldsymbol{\omega}_j + \sum_i \alpha_{ki}^2 \boldsymbol{\omega}_i^2\big)\Big]\\
=& c_k^Q \Big(p_k^2 - \sum_i 2\mu_i \alpha_{ki} p_k + \sum_i \sum_{j \ne i} \alpha_{ki} \alpha_{kj}  \mathbb{E}[\boldsymbol{\omega}_i  \boldsymbol{\omega}_j] + \sum_i \alpha_{ki}^2 \mathbb{E}[\boldsymbol{\omega}_i^2]\Big)\\
=& c_k^Q \Big(p_k^2 - \sum_i 2\mu_i \alpha_{ki} p_k + \sum_i \sum_{j \ne i} \alpha_{ki} \alpha_{kj}  (\mbox{Cov}[\boldsymbol{\omega}_i, \boldsymbol{\omega}_j] + \mu_i \mu_j)\\
 &+ \sum_i \alpha_{ki}^2 (\mbox{Var}[\boldsymbol{\omega}_i] + \mathbb{E}[\boldsymbol{\omega}_i]^2)\Big)\\
=& c_k^Q \Big(p_k^2 - \sum_i 2\mu_i \alpha_{ki} p_k + \sum_i \sum_{j \ne i} \alpha_{ki} \alpha_{kj}  (\kappa_{ij} + \mu_i \mu_j) + \sum_i \alpha_{ki}^2 (\sigma_i^2 + \mu_i^2)\Big)\\
=& c_k^Q \Big((p_k - \sum_i \alpha_{ki} \mu_i)^2 + \sum_i \sum_{j \ne i} \alpha_{ki} \alpha_{kj}  \kappa_{ij} + \sum_i \alpha_{ki}^2 \sigma_i^2 \Big).
\end{align*}
Thus, the expected profit of producer $k$ becomes
\begin{align*}
\Pi_k(p_k, \{\alpha_{ki}\}_{\forall i}) =& \lambda p_k + \sum_i \chi_i \alpha_{ki} - c_k^L\Big(p_k - \sum_i \alpha_{ki} \mu_i\Big)\\
&- c_k^Q \Big(\big(p_k - \sum_i \alpha_{ki} \mu_i\big)^2 + \sum_i \sum_{j \ne i} \alpha_{ki} \alpha_{kj}  \kappa_{ij} + \sum_i \alpha_{ki}^2 \sigma_i^2\Big).
\end{align*}
The first term represents the expected revenue accruing from the sale of electricity. The second term is the expected revenue from the provision of reserves. The third term represents the expected linear cost component. The last term represents the expected quadratic cost component. Note that the quadratic cost component depends on the expected production level, the variance of the forecast errors as well as their covariances. For notational convenience, let $\mu = \begin{pmatrix} \mu_1, \ldots, \mu_N \end{pmatrix}^\top$, $\alpha_k = \begin{pmatrix} \alpha_{k1}, \ldots, \alpha_{kN} \end{pmatrix}^\top$, $\chi = \begin{pmatrix} \chi_1, \ldots, \chi_N \end{pmatrix}^\top$ and let $\Sigma \in \mathbb{R}_+^{N \times N}$ be the covariance matrix of wind forecast errors. Then, the expected profit of producer $k$ can be compactly expressed as
\begin{align*}
\Pi_k(p_k, \alpha_k) =& \lambda p_k + \alpha_k^\top \chi - c_k^L\big(p_k - \alpha_k^\top \mu\big) - c_k^Q \Big(\big(p_k - \alpha_{k}^\top \mu \big)^2 + \alpha_k^\top \Sigma \alpha_k\Big),
\end{align*}
which is a concave function of its variables. Note that the function is strictly concave if $c_k^Q > 0$ and $\sigma_i > 0, \forall i$.

Then, the chance constraints imposed on generation bounds should also be re-formulated to obtain a tractable (hopefully convex) mathematical program. Depending on the underlying forecast error distribution, chance constraints may be recast as convex constraints involving moments of said distribution. In particular, one-sided chance constraints involving a scalar random variable following a Gaussian distribution or approximations of distributionally-robust chance constraints may lead to simple linear inequality constraints \cite{Dvorkin2020}. In the multi-dimensional Gaussian setting, one-sided chance constraints can be re-formulated as second-order cone constraints. In other settings, distributionally-robust chance constraints may be recast as second-order cone constraints \cite{Xie2018}. For simplicity, we assume that $\boldsymbol{\omega}_i \sim \mathcal{N}(\mu_i, \sigma_i^2), \forall i$. The derivation then goes as follows:
\begin{align*}
&\mathbb{P}[\mathbf{p}_k^g(\{\boldsymbol{\omega}_i\}_{\forall i}) \le \overline{p}_k] \ge 1 - \epsilon_k\\
\Leftrightarrow &\mathbb{P}\Big[p_k - \sum_i \alpha_{ki} \boldsymbol{\omega}_i \le \overline{p}_k\Big] \ge 1 - \epsilon_k\\
\Leftrightarrow &\mathbb{P}\Big[\frac{- \sum_i \alpha_{ki} \boldsymbol{\omega}_i + \alpha_k^\top \mu}{\sqrt{\alpha_k^\top \Sigma \alpha_k}} \le \frac{\overline{p}_k - p_k + \alpha_k^\top \mu}{\sqrt{\alpha_k^\top \Sigma \alpha_k}}\Big] \ge 1 - \epsilon_k\\
\Leftrightarrow &\frac{\overline{p}_k - p_k + \alpha_k^\top \mu}{\sqrt{\alpha_k^\top \Sigma \alpha_k}} \ge \Phi^{-1}(1 - \epsilon_k)\\
\Leftrightarrow &p_k \le \overline{p}_k + \alpha_k^\top \mu - ||\alpha_k||_{\Sigma} \Phi^{-1}(1 - \epsilon_k),\\
\end{align*}
which is a second-order cone constraint and where $\Phi^{-1}: (0, 1) \rightarrow \mathbb{R}$ is the quantile function of the standard normal distribution and $||\alpha_k||_{\Sigma} = \sqrt{\alpha_k^\top \Sigma \alpha_k}$. In what follows, we write $\Phi^{-1}(1 - \epsilon_k) = \Phi_k^{-1}$ for brevity. Applying the same ideas to the second chance constraint yields
\begin{align*}
&\mathbb{P}[0 \le \mathbf{p}_k^g(\{\boldsymbol{\omega}_i\}_{\forall i})] \ge 1 - \epsilon_k\\
\Leftrightarrow & \alpha_k^\top \mu + ||\alpha_k||_{\Sigma} \Phi_k^{-1} \le p_k.
\end{align*}
Thus, the deterministic equivalent of the stochastic profit-maximisation problem of producer $k$ reads
\begin{align}
\underset{p_k, \{\alpha_{ki}\}_{\forall i}}{\max} \hspace{10pt} & \lambda p_k + \alpha_k^\top \chi - c_k^L\big(p_k - \alpha_k^\top \mu\big) - c_k^Q \Big(\big(p_k - \alpha_{k}^\top \mu \big)^2 + \alpha_k^\top \Sigma \alpha_k\Big)\\
\mbox{s.t. } & p_k \le \overline{p}_k + \alpha_k^\top \mu - ||\alpha_k||_{\Sigma} \Phi_k^{-1}, \hspace{25pt} (\overline{\nu}_k)\\
& \alpha_k^\top \mu + ||\alpha_k||_{\Sigma} \Phi_k^{-1} \le p_k, \hspace{48pt}(\underline{\nu}_k)\\
& 0 \le \alpha_{ki} \le 1, \forall i,\\
&p_k \in \mathbb{R},
\end{align}
which is a quadratically-constrained (concave) quadratic program.
\subsection{Market Operator}

\subsubsection{Stochastic Formulation}

The market operator seeks to clear the energy market and procure enough reserves to cover any forecast error,
\begin{align}
& \sum_k \mathbf{p}_k^g(\{\boldsymbol{\omega}_i\}_{\forall i}) + \sum_i \mathbf{p}_i^w(\boldsymbol{\omega}_i) = D,\\
& \sum_{k} \alpha_{ki} = 1, \forall i.
\end{align}

\subsubsection{Deterministic Equivalent Formulation}

Simply substituting the affine control law of flexible generators yields
\begin{align}
& \sum_k p_k + \sum_i \tilde{W}_i = D, \hspace{10pt} (\lambda)\\
& \sum_k \alpha_{ki} = 1, \forall i, \hspace{35pt} (\chi_i)
\end{align}
where electricity and reserve prices are obtained as the dual variables of the power balance and reserve allocation constraints.
\subsection{Wind Producers}

Wind producers are price takers and do not control their output (i.e., no spillage is allowed), and therefore do not solve any optimisation problem \textit{per se}. It is assumed that wind producers are paid for their forecast production and are charged for the uncertainty that they introduce in the system. More precisely, this charge is proportional to the dual variable of the reserve allocation constraint associated with a given wind producer. Their expected revenue can thus be computed as $R = \mathbb{E}[\lambda \tilde{W}_i - \chi_i] = \lambda \tilde{W}_i - \chi_i$.

\subsection{Electricity Consumers}

The demand is assumed to be inelastic and electricity consumers do not solve any optimisation problem either. Their expected pay-off can be calculated as $P = \mathbb{E}[\lambda D] = \lambda D$.

\section{Market Clearing Problem Formulation}

This section describes a stochastic market clearing problem that can be solved by a market operator in order to identify prices for energy and reserves, schedule generators and allocate reserves in a cost-efficient way. 

\subsection{Stochastic Formulation}
Under the assumption that the market operator is risk-neutral, the stochastic market clearing problem can be formulated as

\begin{align}
\underset{\{p_k, \{\alpha_{ki}\}_{\forall i}\}_{\forall k}}{\min} \hspace{10pt} & \mathbb{E}\Big[\sum_k \big(c_k^L \mathbf{p}_k^g(\{\boldsymbol{\omega}_i\}_{\forall i}) + c_k^Q \big(\mathbf{p}_k^g(\{\boldsymbol{\omega}_i\}_{\forall i})\big)^2\big)\Big]\\
\mbox{s.t. } & \mathbb{P}[\mathbf{p}_k^g(\{\boldsymbol{\omega}_i\}_{\forall i}) \le \overline{p}_k] \ge 1 - \epsilon_k, \mbox{ }\forall k,\\
& \mathbb{P}[0 \le \mathbf{p}_k^g(\{\boldsymbol{\omega}_i\}_{\forall i})] \ge 1 - \epsilon_k, \mbox{ }\forall k,\\
& \sum_k \mathbf{p}_k^g(\{\boldsymbol{\omega}_i\}_{\forall i}) + \sum_i \mathbf{p}_i^w(\boldsymbol{\omega}_i) = D,\\
& \sum_k \alpha_{ki} = 1, \forall i,\\
& 0 \le \alpha_{ki} \le 1, \forall k, \forall i,\\
& p_k \in \mathbb{R}, \forall k.
\end{align}

\subsection{Deterministic Equivalent Formulation}
Using techniques introduced earlier yields the following equivalent deterministic formulation of the market clearing problem:
\begin{align}
\underset{\{p_k, \alpha_k\}_{\forall k}}{\min} \hspace{10pt} & \sum_k \Big(c_k^L\big(p_k - \alpha_k^\top \mu\big) - c_k^Q \Big(\big(p_k - \alpha_{k}^\top \mu \big)^2 + \alpha_k^\top \Sigma \alpha_k\Big)\Big)\\
\mbox{s.t. } & p_k \le \overline{p}_k + \alpha_k^\top \mu - ||\alpha_k||_{\Sigma} \Phi_k^{-1}, \mbox{ }\forall k, \hspace{15pt}(\overline{\nu}_k)\\
& \alpha_k^\top \mu + ||\alpha_k||_{\Sigma} \Phi_k^{-1} \le p_k, \mbox{ }\forall k, \hspace{37pt}(\underline{\nu}_k)\\
& \sum_k p_k + \sum_i \tilde{W}_i = D, \hspace{70pt} (\lambda)\\
& \sum_k \alpha_{ki} = 1, \forall i, \hspace{95pt} (\chi_i)\\
& 0 \le \alpha_{ki} \le 1, \forall k, \forall i,\\
& p_k \in \mathbb{R}, \forall k,
\end{align}
where the dual variables of the energy balance and reserve allocation constraints are used to set energy and reserve prices, respectively. Note that computing expectations and treating chance constraints in the problem solved by the market operator in the same way as in the problem of a market participant is only possible under the assumption that they all share the same information about $\boldsymbol{\omega}$. It will become apparent later that this is also a pre-requisite for the pricing scheme of the market operator to support a competitive equilibrium (since the complementarity problems stemming from the equilibrium and market clearing approaches would otherwise be different). This topic is also discussed in the context of a stochastic programming-based market design by Dvorkin et al. \cite{DvorkinV2019}.

\section{Market Properties}

We analyse four key properties of the proposed stochastic market design, namely whether it supports a \textit{competitive equilibrium}, guarantees \textit{cost recovery} for flexible generators, \textit{revenue adequacy} for the market operator, and is \textit{incentive compatible}.

\begin{definition}
(Competitive Equilibrium) A competitive equilibrium for the stochastic market is a set of prices $\{\lambda^*, \{\chi_i^*\}_{\forall i}\}$ and decisions $\{p_k^*, \{\alpha_{ki}^*\}_{\forall i}\}_{\forall k}$ that\vspace{-5pt}
\begin{enumerate}
\item clear the market: $\sum_k p_k^* + \sum_i \tilde{W}_i = D$ and $\sum_k \alpha_{ki}^* = 1, \forall i$\vspace{-5pt}
\item maximise the expected profit of flexible generators
\end{enumerate}
\end{definition}

\begin{proposition}
(\textcolor{orange}{Competitive Equilibrium}) The stochastic market clearing problem produces prices and decisions maximising the profit of each flexible generator and supporting a competitive equilibrium from which they have no incentive to deviate.
\end{proposition}
\begin{proof}
The proof showing that decisions $\{p_k^*, \alpha_k^*\}$ and prices $\{\lambda^*, \chi^*\}$ support a competitive equilibrium proceeds in two steps. The first step consists in finding an expression for the profit earned by each producer for prices and decisions computed by the market operator. The second step consists in showing that this profit is optimal for the problem faced by each producer.

\textit{First Step:} Let $V_k^*$ denote the value of the objective of producer $k$ under said market prices and decisions
\begin{align*}
    V_k^* =& \lambda^*p_k^* + \alpha_k^*\chi^* - c_k^L(p_k^* - \alpha_k^* \mu) - c_k^Q\big((p_k^* - \alpha_k^* \mu)^2 + (\alpha_k^*)^2\sigma^2\big) \\
    =& \lambda^*p_k^* + \alpha_k^*\chi^* - c_k^L(p_k^* - \alpha_k^* \mu) - c_k^Q\big((p_k^* - \alpha_k^* \mu)^2 + (\alpha_k^*)^2\sigma^2\big)\\
    & + \underline{\nu}_k^* (p_k^* - \alpha_k^* \phi_k) + \overline{\nu}_k^*(-p_k^* - \alpha_k^* \phi_k + \bar{p}_k)\\
    =& p_k^*\big(\lambda^* - c_k^L + \underline{\nu}_k^* - \overline{\nu}_k^*\big) - c_k^Q(p_k^* - \alpha_k^* \mu)^2 + \overline{\nu}_k^* \overline{p}_k\\
    &+ \alpha_k^*\big(\chi^* + c_k^L \mu - \underline{\nu}_k^* \phi_k - \overline{\nu}_k^* \phi_k\big) - c_k^Q (\alpha_k^*)^2 \sigma^2 \\
    =& p_k^*\big(2c_k^Q(p_k^* - \alpha_k^* \mu)\big) - c_k^Q(p_k^* - \alpha_k^* \mu)^2 + \overline{\nu}_k^* \overline{p}_k\\ 
    &+ \alpha_k^*\big(2c_k^Q(\alpha_k^*(\mu^2+\sigma^2) - p_k^* \mu)\big) - c_k^Q (\alpha_k^*)^2 \sigma^2\\
    =& 2c_k^Q(p_k^* - \alpha_k^* \mu)^2 + 2c_k^Q (\alpha_k^*)^2 \sigma^2 + \overline{\nu}_k^* \overline{p}_k - c_k^Q(p_k^* - \alpha_k^* \mu)^2 - c_k^Q (\alpha_k^*)^2 \sigma^2\\
    =& c_k^Q(p_k^* - \alpha_k^* \mu)^2 + c_k^Q (\alpha_k^*)^2 \sigma^2 + \overline{\nu}_k^* \overline{p}_k\\
    \ge& 0
\end{align*}
The second line results from the fact that the two new terms added to the objective are equal to zero (complementary slackness). The third line is obtained by re-arranging terms. The fourth line follows from the stationarity of the Lagrangian with respect to $p_k$ and $\alpha_k$, respectively (see KKT conditions in Appendix A). The fifth and sixth lines follow from re-arranging and cancelling terms out. The last line follows from the fact that the first two terms are non-negative (quadratic terms with positive coefficient), and the third term is non-negative too, since $\overline{\nu}_k^* \ge 0$ (dual feasibility) and $\overline{p}_k > 0$ (by assumption). Note that this result also amounts to proving that the cost recovery property holds in expectation (since the expected revenue earned by producer $k$ is greater than or equal to the expected costs she incurs).

\textit{Second Step:} Now, let $\{p_k^\star, \alpha_k^\star\}$ denote the solution of the problem faced by producer $k$ under prices $\{\lambda^*, \chi^*\}$, and let $V_k^\star$ denote the corresponding objective. We essentially seek to show that $V_k^\star \le V_k^*$. One successively finds
\begin{align*}
    V_k^\star =& \lambda^*p_k^\star + \chi^* \alpha_k^\star - c_k^L(p_k^\star - \alpha_k^\star \mu) - c_k^Q\big((p_k^\star - \alpha_k^\star \mu)^2 + (\alpha_k^\star)^2\sigma^2\big)\\
    \le& \lambda^*p_k^\star + \chi^* \alpha_k^\star - c_k^L(p_k^\star - \alpha_k^\star \mu) - c_k^Q\big((p_k^\star - \alpha_k^\star \mu)^2 + (\alpha_k^\star)^2\sigma^2\big)\\
    &+ \underline{\nu}_k^* (p_k^\star - \alpha_k^\star \phi_k) + \overline{\nu}_k^*(-p_k^\star - \alpha_k^\star \phi_k + \bar{p}_k)\\
    =& p_k^\star\big(\lambda^* - c_k^L + \underline{\nu}_k^* - \overline{\nu}_k^*\big) - c_k^Q(p_k^\star - \alpha_k^\star \mu)^2 + \overline{\nu}_k^* \overline{p}_k\\
    &+ \alpha_k^\star\big(\chi^* + c_k^L \mu - \underline{\nu}_k^* \phi_k - \overline{\nu}_k^* \phi_k\big) - c_k^Q (\alpha_k^\star)^2 \sigma^2 \\
   =& p_k^\star\big(2c_k^Q(p_k^* - \alpha_k^* \mu)\big) - c_k^Q(p_k^\star - \alpha_k^\star \mu)^2 + \overline{\nu}_k^* \overline{p}_k\\ 
    &+ \alpha_k^\star\big(2c_k^Q(\alpha_k^*(\mu^2+\sigma^2) - p_k^* \mu)\big) - c_k^Q (\alpha_k^\star)^2 \sigma^2,
\end{align*}
where the second line follows from the fact that $\underline{\nu}_k^* \ge 0$ and $\overline{\nu}_k^* \ge 0$ (dual feasibility for market operator), while the terms in parentheses multiplying them are non-negative (primal feasibility of ($p_k^\star, \alpha_k^\star$) for producer $k$). The third line is obtained by re-arranging terms and the fourth line follows from the stationarity of the Lagrangian of the market clearing problem with respect to $p_k$ and $\alpha_k$ (see KKT conditions in Appendix A). Let $f:\mathbb{R}^2 \rightarrow \mathbb{R}$ be such that
\begin{align*}
f(x, y) =& x \big(2c_k^Q(p_k^* - \alpha_k^* \mu)\big) - c_k^Q(x - y \mu)^2 + \overline{\nu}_k^* \overline{p}_k\\ 
    &+ y \big(2c_k^Q(\alpha_k^*(\mu^2+\sigma^2) - p_k^* \mu)\big) - c_k^Q y^2 \sigma^2.
\end{align*}
We first note that
\begin{align*}
V_k^\star \le f(p_k^\star, \alpha_k^\star),
\end{align*} 
and
\begin{equation*}
f(p_k^*, \alpha_k^*) = V_k^*,
\end{equation*}
which directly follows from the derivations in the first step of this proof. We now seek to show that $f(p_k^\star, \alpha_k^\star) \le f(p_k^*, \alpha_k^*)$. To achieve this, we proceed in two steps. First, we check that $(x, y) = (p_k^*, \alpha_k^*)$ is a stationary point of $f$. Then, we check that it also corresponds to a maximum of $f$ by studying the properties of its Hessian matrix. We first compute
\begin{align*}
\frac{\partial f}{\partial x} &= 2c_k^Q(p_k^* - \alpha_k^*\mu) - 2 c_k^Q x + 2c_k^Q y \mu,\\
\frac{\partial f}{\partial y} &= - 2c_k^Q y \mu^2 + 2 c_k^Q x \mu + 2c_k^Q\big(\alpha_k^*(\mu^2 + \sigma^2) - p_k^* \mu\big) - 2c_k^Q y \sigma^2,
\end{align*}
from which it is clear that setting $(x, y) = (p_k^*, \alpha_k^*)$ cancels both derivatives. We then derive an expression for the Hessian,
\begin{equation*}
H(x,y) = \begin{pmatrix} \frac{\partial^2 f}{\partial x^2} & \frac{\partial^2 f}{\partial y \partial x} \\ \frac{\partial^2 f}{\partial x \partial y} & \frac{\partial^2 f}{\partial y^2} \end{pmatrix} = 2c_k^Q \begin{pmatrix} -1 & \mu \\ \mu & -\mu^2 - \sigma^2 \end{pmatrix}.
\end{equation*}
We note that the first entry on the diagonal is negative. In addition, we have that $\mbox{det}|H(p_k^*, \alpha_k^*)| = 2 c_k^Q \sigma^2 > 0$ (unless $\sigma = 0$), which in turn implies that the Hessian is negative definite. This result has two further implications. First, by virtue of the second-derivative test, $(x, y) = (p_k^*, \alpha_k^*)$ is a local maximum of $f$. Second, since the Hessian is negative definite and independent of $(x, y)$, $f$ is concave and $(x, y) = (p_k^*, \alpha_k^*)$ is a global maximum. In other words, $f(x, y) \le f(p_k^*, \alpha_k^*), \forall (x, y),$ and in particular $ f(p_k^\star, \alpha_k^\star) \le f(p_k^*, \alpha_k^*)$. It then follows that
\begin{equation*}
V_k^\star \le f(p_k^\star, \alpha_k^\star) \le f(p_k^*, \alpha_k^*) = V_k^*.
\end{equation*}
Since this applies to any arbitrary producer $k$, $V_k^\star \le V_k^*, \forall k$. Prices and decisions identified by the market operator therefore maximise the profit of each producer, resulting in a competitive equilibrium.
\end{proof}

\begin{proposition}
(\textcolor{orange}{Cost Recovery}) The stochastic market clearing problem produces prices and decisions that guarantee a non-negative expected pay-off for flexible generators.
\end{proposition}
\begin{proof}
We first re-write the problem faced by producer $k$ in the standard QP form introduced in Appendix B,
\begin{align*}
\underset{x}{\min} \hspace{10pt} & \frac{1}{2}x^\top Q x + c^\top x\\
\mbox{s.t. } & Ax \le b \hspace{15pt} (\pi)\\
&x \in \mathbb{R}^n,
\end{align*}
where
\begin{equation*}
x = \begin{pmatrix} p_k \\ \alpha_k \end{pmatrix}, c = \begin{pmatrix} c_k^L - \lambda \\ -c_k^L \mu - \chi \end{pmatrix}, A = \begin{pmatrix} 1 & \phi_k \\ -1 & \phi_k \end{pmatrix}, b = \begin{pmatrix} \overline{p}_k\\0\end{pmatrix}, \pi = \begin{pmatrix} -\overline{\nu}_k \\ -\underline{\nu}_k \end{pmatrix}.
\end{equation*}
Note that the entries of $\pi$ must be non-positive (following the convention used in Appendix B), and a minus sign must therefore be added in front of $\overline{\nu}_k$ and $\underline{\nu}_k$. In addition, the bounds on $\alpha_k$ do not impact developments below and were neglected for conciseness. The Hessian of the objective and its inverse read
\begin{equation*}
Q = 2 c_k^Q  \begin{pmatrix} 1 & -\mu \\ -\mu & \mu^2 + \sigma^2 \end{pmatrix} \mbox{ and } Q^{-1} = \frac{1}{2 c_k^Q \sigma^2} \begin{pmatrix} \mu^2 + \sigma^2 & \mu \\ \mu & 1 \end{pmatrix},
\end{equation*}
where $Q^{-1}$ can be obtained by Cramer's rule. By Sylvester's criterion, $Q$ is positive definite. Indeed its leading principal minors, which are equal to 
\begin{equation*}
2c_k^Q \mbox{ and } 2c_k^Q\sigma^2,
\end{equation*}
are both positive (unless $\sigma = 0$). It also follows that $Q^{-1}$ is positive definite. As shown in Appendix B, the dual problem of producer $k$ reads 
\begin{align*}
\underset{\pi}{\max} \hspace{10pt} & -\frac{1}{2}(A^\top \pi -c)^\top Q^{-1} (A^\top \pi -c) + \pi^\top b\\
\mbox{s.t. } & \pi \le 0.
\end{align*}
Note that the first term in the dual objective is non-positive, since $Q^{-1}$ is positive definite. In addition,
\begin{equation*}
\pi^\top b = \begin{pmatrix}-\overline{\nu}_k & -\underline{\nu}_k \end{pmatrix}\begin{pmatrix} \overline{p}_k\\0\end{pmatrix} = -\overline{\nu}_k \overline{p}_k \le 0,
\end{equation*}
as $\overline{\nu}_k \ge 0$ for any dual feasible solution and it is assumed that $\overline{p}_k > 0$. The optimal dual objective is therefore non-positive. Then, strong duality for convex quadratic programs \cite{Dorn1960} implies that the primal and dual objectives are equal at optimality. Let $(p_k^*, \alpha_k^*)$ be an optimal primal solution. It follows that
\begin{equation*}
c_k^L(p_k^* - \alpha_k^* \mu) + c_k^Q \big((p_k^* - \alpha_k^* \mu)^2 + (\alpha_k^*)^2 \sigma^2\big) - \lambda p_k^* - \chi \alpha_k^* \le 0,
\end{equation*}
which in turn implies that
\begin{align*}
\mathbb{E}[\lambda p_k^* + \chi \alpha_k^*] &= \lambda p_k^* + \chi \alpha_k^*\\
&\ge c_k^L(p_k^* - \alpha_k^* \mu) + c_k^Q \big((p_k^* - \alpha_k^* \mu)^2 + (\alpha_k^*)^2 \sigma^2\big)\\ 
&= \mathbb{E}[c_k^L(p_k^* - \alpha_k^* \boldsymbol{\omega}) + c_k^Q (p_k^* - \alpha_k^* \boldsymbol{\omega})^2],
\end{align*}
or, alternatively,
\begin{equation*}
\mathbb{E}[\lambda p_k^* + \chi \alpha_k^* - c_k^L(p_k^* - \alpha_k^* \boldsymbol{\omega}) - c_k^Q (p_k^* - \alpha_k^* \boldsymbol{\omega})^2] \ge 0.
\end{equation*}
In other words, the expected revenue of producer $k$ is greater than or equal to its expected cost. The cost recovery property is therefore satisfied in expectation.

This property can also be proved (more concisely) using the complementary slackness conditions of the problem faced by the producer (in a fashion similar to the first step of the proof of the competitive equilibrium property above). 
\end{proof}

\begin{proposition}
(\textcolor{green}{Revenue Adequacy}) The stochastic market clearing problem produces prices and decisions guaranteeing that the market operator does not incur any financial loss.
\end{proposition}
\begin{proof}
Let $\{\lambda^*, \{\chi_i^*\}_{\forall i}\}$ and $\{p_k^*, \{\alpha_{ki}^*\}_{\forall i}\}_{\forall k}$ denote prices and decisions calculated by the market operator. Consumers only pay the market operator for the electricity consumed, while flexible generators receive payments covering energy and reserves. Wind producers, on the other hand, are paid for their energy and charged for the variability they introduce into the system. More precisely, if wind producers are paid for their forecast production $\tilde{W}_i$ and charged for their variability via the reserve price $\chi_i$, while flexible generators are paid for their scheduled production $p_k^*$ and contribution to reserve procurement $\{\alpha_{ki}^*\}_{\forall i}$, respectively, the surplus collected by the market operator is as follows,
\begin{align*}
\Delta^* &= \lambda^*D + \sum_i \chi_i^* - \sum_i \lambda^*\tilde{W}_i - \sum_k \lambda^*p_k^* - \sum_k \sum_i \alpha_{ki}^* \chi_i^*\\
&= \lambda^*\big(D - \sum_i \tilde{W}_i - \sum_k p_k^*\big) + \sum_i \chi_i (1 - \sum_k \alpha_{ki}^*)\\
&= 0,
\end{align*}
where the third line follows the fact that the markets for energy and reserves clear (primal feasibility of market clearing problem). Note that this result holds for any forecast error realisation.
\end{proof}

\begin{proposition}
(\textcolor{red}{Incentive Compatibility}) The stochastic market clearing problem does not produce prices and decisions guaranteeing that each flexible generator maximises its profit by bidding truthfully.
\end{proposition}
\begin{proof}
By contradiction. Assume that incentive compatibility holds and see counter-example implemented in Julia \cite{SMER2022}, where a producer operating in an electricity market with a finite number of producers can increase its profit by bidding untruthfully.
\end{proof}

Market properties are also discussed in \cite{Ratha2019} in a similar set-up. The proofs of incentive compatibility and revenue adequacy seem weird, however. The proof of cost recovery looks weird too (specifically the fact that the dual problem is a linear program, in spite of the fact that the primal problem is a quadratic program). 

\section{Sensitivity Analysis}

In this section, we carry out a sensitivity analysis of the stochastic market clearing problem with respect to two essential statistical parameters, namely the mean $\mu$ and standard deviation $\sigma$ of the forecast error.

Let $(\{p_k^*\}_{\forall k}, \{\alpha_k^*\}_{\forall k})$ and $(\lambda^*, \chi^*, \{\underline{\nu}_k^*\}_{\forall k}, \{\overline{\nu}_k^*\}_{\forall k})$ denote optimal primal and dual solutions obtained by solving the stochastic market clearing problem. Then, let $V^*: \mathbb{R} \times \mathbb{R}_+ \rightarrow \mathbb{R}$ denote the value function of the stochastic market clearing problem, that is, 
\begin{align*}
V^*(\mu, \sigma) &= \mathbb{E}\Big[\sum_k \big(c_k^L \mathbf{p}_k^*(\boldsymbol{\omega}) + c_k^Q (\mathbf{p}_k^*(\boldsymbol{\omega}))^2\big)\Big]\\
 &= \sum_k \Big(c_k^L(p_k^* - \alpha_k^* \mu) + c_k^Q \big((p_k^* - \alpha_k^* \mu)^2 + (\alpha_k^*)^2 \sigma^2\big)\Big).
\end{align*}
In the following, we assume that we work with parameter values $\mu$ and $\sigma$ such that the value function can always be defined (i.e., the optimal solution set is non-empty). Before presenting our main result, we prove the following lemma.
\begin{lemma}\label{lemma1}
Let $V^*$ be the value function of the market clearing problem. Then, if $\sigma > 0$ and $c_k^Q > 0, \forall k$, we have that
\begin{align*}
\frac{\partial V^*}{\partial \mu} &= \frac{\partial \mathcal{L}^*}{\partial \mu},\\
\frac{\partial V^*}{\partial \sigma} &= \frac{\partial \mathcal{L}^*}{\partial \sigma},
\end{align*}
where right-hand side expressions should be interpreted as the partial derivatives of the Lagrangian function $\mathcal{L}$ (whose expression is given in Appendix A) evaluated at an optimal solution.
\end{lemma}
\begin{proof}
The proof relies on an enveloppe theorem developed for saddle point and constrained convex programming problems \cite{Milgrom2002}. In essence, this theorem (Theorem 5 and Corollary 5 in \cite{Milgrom2002}) states that if i) the set over which optimisation takes place is convex and compact, ii) the functions defining the objective and constraints are continuous, iii) the derivatives of the functions forming the constraints taken with respect to the parameters are continuous (both in the variables and parameters), iv) some constraint qualification holds (e.g., Slater's condition is satisfied), then the value function is directionally differentiable and can be expressed in terms of the derivative of the Lagrangian function (with respect to parameters). In addition, if the set of optimal solutions corresponding to given parameters is a singleton, the value function is differentiable at these parameters and it is equal to the derivative of the Lagrangian function evaluated at the optimal solution.

First, note that unlike the stochastic market clearing problem, the problem for which the result in \cite{Milgrom2002} was derived does not include any equality constraints. It is nonetheless possible to define an equivalent market clearing formulation featuring only inequality constraints by eliminating equality constraints (e.g., via the QR decomposition of the equality constraint matrix) and optimising directly over the resulting affine space. Note that an affine space is also convex, and a compact subset of the affine space containing the optimal solution associated with given parameters can be considered for our purpose. Hence, the market clearing problem satisfies conditions i), ii), iii) and iv). In particular, it is possible to find feasible solutions strictly satisfying inequality constraints, such that Slater's condition is satisfied. Now, since $c_k^Q > 0, \forall k,$ and $\sigma > 0$, the Hessian of the objective function is positive definite, which in turn implies that the objective of the market clearing problem is strictly convex and the problem therefore has a unique optimal solution. Thus, by virtue of Corollary 5 in \cite{Milgrom2002}, the partial derivatives of the value function with respect to $\mu$ and $\sigma$ can be expressed as the partial derivatives of the Lagrangian function evaluated at the corresponding optimal solution.
\end{proof}
Next, we have the following result.
\begin{proposition}
(Sensitivity of Expected System Cost to Forecast Error) The reserve price measures the combined sensitivity of the expected system cost with respect to the mean and standard deviation of the forecast error, such that
\begin{equation*}
\chi^* = \mu \frac{\partial V^*}{\partial \mu} + \sigma \frac{\partial V^*}{\partial \sigma}.
\end{equation*}
\end{proposition}
\begin{proof}
Building upon Lemma \ref{lemma1}, we successively have
\begin{align*} 
\frac{\partial V^*}{\partial \mu} &= \frac{\partial \mathcal{L}^*}{\partial \mu}\\
&= \sum_k\big(-c_k^L \alpha_k^* - 2c_k^Q \alpha_k^* p_k^* + 2 c_k^Q (\alpha_k^*)^2 \mu - \underline{\nu}_k^* \alpha_k^* - \overline{\nu}_k^* \alpha_k \big)\\
&= \sum_k\alpha_k^*\big(-c_k^L-2c_k^Q(p_k^*-\alpha_k^* \mu)-\underline{\nu}_k^*-\overline{\nu}_k^*\big)\\
&= \sum_k\alpha_k^*\Big(\frac{\chi^*}{\mu} - \Phi^{-1}(1-\epsilon_k) \frac{\sigma}{\mu}(\underline{\nu}_k^*+\overline{\nu}_k^*) - 2c_k^Q \alpha_k^* \frac{\sigma^2}{\mu}\Big)\\
&= \sum_k\frac{\alpha_k^*}{\mu}\Big(\chi^* - \Phi^{-1}(1-\epsilon_k) \sigma(\underline{\nu}_k^*+\overline{\nu}_k^*) - 2c_k^Q \alpha_k^* \sigma^2\Big),
\end{align*}
where the first line follows from Lemma \ref{lemma1}, the second line is obtained by taking the partial derivative of the Lagrangian function with respect to $\mu$ and evaluating it at the optimal solution, re-arranging terms yields the third line, the fourth line is obtained by substituting the stationarity condition of the Lagrangian function with respect to $\alpha_k$ (satisfied at optimality), while re-arranging terms gives the fifth lines. It immediately follows that
\begin{equation*}
\mu \frac{\partial V^*}{\partial \mu} = \sum_k\alpha_k^*\Big(\chi^* - \Phi^{-1}(1-\epsilon_k) \sigma(\underline{\nu}_k^*+\overline{\nu}_k^*) - 2c_k^Q \alpha_k^* \sigma^2\Big).
\end{equation*}
Then, we successively have
\begin{align*}
\frac{\partial V^*}{\partial \sigma} &= \frac{\partial \mathcal{L}^*}{\partial \sigma}\\
&= \sum_k\big(2c_k^Q(\alpha_k^*)^2\sigma + \Phi^{-1}(1-\epsilon_k) \alpha_k^* \underline{\nu}_k^* + \Phi^{-1}(1-\epsilon_k) \alpha_k^* \overline{\nu}_k^*\big)\\
&=\sum_k\frac{\alpha_k^*}{\sigma}\Big(2c_k^Q\alpha_k^*\sigma^2 + \Phi^{-1}(1-\epsilon_k) \sigma (\underline{\nu}_k^* + \overline{\nu}_k^*)\Big),
\end{align*}
where the first line follows from Lemma \ref{lemma1}, the second line is obtained by taking the partial derivative of the Lagrangian function with respect to $\sigma$ and evaluating it at the optimal solution, while re-arranging terms yields the third line. In other words, we get
\begin{equation*}
\sigma \frac{\partial V^*}{\partial \sigma} = \sum_k \alpha_k^*\Big(2c_k^Q\alpha_k^*\sigma^2 + \Phi^{-1}(1-\epsilon_k) \sigma (\underline{\nu}_k^* + \overline{\nu}_k^*)\Big).
\end{equation*}
Combining these two results then yields
\begin{align*}
\mu \frac{\partial V^*}{\partial \mu} &= \sum_k\alpha_k^*\Big(\chi^* - \Phi^{-1}(1-\epsilon_k) \sigma(\underline{\nu}_k^*+\overline{\nu}_k^*) - 2c_k^Q \alpha_k^* \sigma^2\Big)\\
&= \sum_k\alpha_k^*\chi^* - \sigma \frac{\partial V^*}{\partial \sigma}.
\end{align*}
Since all contributions $\{\alpha_k^*\}_{\forall k}$ to reserves sum to 1, we get the desired result
\begin{equation*}
\mu \frac{\partial V^*}{\partial \mu} + \sigma \frac{\partial V^*}{\partial \sigma} = \chi^*.
\end{equation*}
\end{proof}
We also have the following result.
\begin{proposition}
The expected system cost increases with the standard deviation of the forecast error.
\end{proposition}
\begin{proof}
It follows from the proof of the previous proposition that
\begin{equation*}
\frac{\partial V^*}{\partial \sigma} = \sum_k\frac{\alpha_k^*}{\sigma}\Big(2c_k^Q\alpha_k^*\sigma^2 + \Phi^{-1}(1-\epsilon_k) \sigma (\underline{\nu}_k^* + \overline{\nu}_k^*)\Big).
\end{equation*}
Note that $\alpha_k^* \ge 0$, $\underline{\nu}_k^* \ge 0$, $\overline{\nu}_k^* \ge 0$, $\forall k$. By definition, $\sigma \ge 0$, $c_k^Q \ge 0$ and $\epsilon_k \le 0.5$, which implies that $\Phi^{-1}(1-\epsilon_k) \ge 0$. Hence, we have
\begin{equation*}
\frac{\partial V^*}{\partial \sigma} \ge 0.
\end{equation*}
\end{proof}

%\section{Comments}
%
%\begin{itemize} 
%\item It is assumed that flexible generators get paid for the scheduled production $p_k$ in this document. Interestingly, if they get paid for the (expected) real-time production $\mathbb{E}[\mathbf{p}_k(\boldsymbol{\omega})] = p_k - \alpha_k \mu$ instead, the complementarity problems associated with the equilibrium problem and the market clearing problem are different (an additional term  appears in the $\alpha_k$ stationarity condition of producer $k$)
%\item Likewise, it was assumed that wind producers get paid for the forecast production $\tilde{W}$, but assuming that they get paid for the expected production $\mathbb{E}[\mathbf{p}_w(\boldsymbol{\omega})] = \tilde{W} + \mu$ could have an impact on strategic behaviour.
%\end{itemize}

\section*{Appendix A}
This section derives complementary problems based on the KKT conditions of the deterministic equivalent of the equilibrium and market problems and shows that they are identical (hence the equivalence between solutions of the two problems.

\subsection*{Equilibrium Problem}
To ease the writing of KKT conditions, the problem faced by producer $k$ is first re-written as a minimisation problem,
\begin{align}
-\underset{p_k, \{\alpha_{ki}\}_{\forall i}}{\min} \hspace{10pt} &  c_k^L\big(p_k - \alpha_k^\top \mu\big) + c_k^Q \Big(\big(p_k - \alpha_{k}^\top \mu \big)^2 + \alpha_k^\top \Sigma \alpha_k\Big) - \lambda p_k - \alpha_k^\top \chi\\
\mbox{s.t. } & p_k \le \overline{p}_k + \alpha_k^\top \mu - ||\alpha_k||_{\Sigma} \Phi_k^{-1}, \hspace{25pt} (\overline{\nu}_k)\\
& \alpha_k^\top \mu + ||\alpha_k||_{\Sigma}\Phi_k^{-1} \le p_k, \hspace{48pt}(\underline{\nu}_k)\\
& 0 \le \alpha_{ki} \le 1, \forall i,\\
& p_k \in \mathbb{R}.
\end{align}
Then, the Lagrangian function of producer $k$ can be expressed as
\begin{align*}
\mathcal{L}_k(p_k, \alpha_k, \lambda, \chi, \underline{\nu}_k, \overline{\nu}_k) =& c_k^L(p_k - \alpha_k^\top \mu) + c_k^Q \big((p_k - \alpha_k^\top \mu)^2 + \alpha_k^\top \Sigma \alpha_k\big)\\
&- \lambda p_k - \alpha_k^\top \chi + \underline{\nu}_k(\alpha_k^\top \mu + ||\alpha_k||_{\Sigma} \Phi_k^{-1} - p_k)\\
&+ \overline{\nu}_k (p_k - \overline{p}_k - \alpha_k^\top \mu + ||\alpha_k||_{\Sigma} \Phi_k^{-1}).
\end{align*}
The KKT conditions of the problem faced by producer $k$ now follow
\begin{align*}
\frac{\partial \mathcal{L}_k}{\partial p_k} =& c_k^L + 2 c_k^Q (p_k - \alpha_k^\top \mu) - \lambda - \underline{\nu}_k + \overline{\nu}_k = 0, \forall k,\\
\frac{\partial \mathcal{L}_k}{\partial \alpha_{ki}} =& - c_k^L \mu_i + 2 c_k^Q\Big(\alpha_{ki}\mu_i^2 - p_k \mu_i + \sigma_i^2 \alpha_{ki} + \sum_{i, j>i} \kappa_{ij} \alpha_{kj}\Big) - \chi_i \\
&+ \underline{\nu}_k \Big(\mu_i + \Phi_k^{-1} \frac{\sigma_i^2 \alpha_{ki} + \sum_{i, j>i} \kappa_{ij} \alpha_{kj}}{||\alpha_k||_{\Sigma}}\Big)\\
&+ \overline{\nu}_k \Big(\mu_i + \Phi_k^{-1} \frac{\sigma_i^2 \alpha_{ki} + \sum_{i, j>i} \kappa_{ij} \alpha_{kj}}{||\alpha_k||_{\Sigma}}\Big) = 0, \forall k, \forall i,\\
0 \le \overline{\nu}_k &\perp \overline{p}_k + \alpha_k^\top \mu - ||\alpha_k||_{\Sigma} \Phi_k^{-1} - p_k \ge 0,\\
0 \le \underline{\nu}_k & \perp p_k - \alpha_k^\top \mu - ||\alpha_k||_{\Sigma}\Phi_k^{-1} \ge 0,
\end{align*}
where the range constraints of $\alpha_{ki}$ variables were omitted for conciseness. Gathering the KKT conditions of each producer and adding the market clearing constraints yields the complementarity problem associated with the equilibrium problem.

\subsection*{Stochastic Market Clearing Problem}
For simplicity, let 
\begin{align*}
x &= \begin{pmatrix} \{p_k\}_{\forall k}, \{\alpha_k\}_{\forall k} \end{pmatrix}^\top,\\
\pi &= \begin{pmatrix} \lambda, \chi, \{\underline{\nu}_k\}_{\forall k}, \{\overline{\nu}_k\}_{\forall k} \end{pmatrix}^\top,
\end{align*}
denote the set of primal and dual variables, respectively. Let us form the Lagrangian function of the stochastic market clearing problem:
\begin{align*}
\mathcal{L}(x, \pi) =& \sum_k \Big(c_k^L(p_k - \alpha_k^\top \mu) + c_k^Q \big((p_k - \alpha_k^\top \mu)^2 + \alpha_k^\top \Sigma \alpha_k\big)\Big)\\
&+ \sum_k \underline{\nu}_k(\alpha_k^\top \mu + ||\alpha_k||_{\Sigma} \Phi_k^{-1} - p_k)\\
&+ \sum_k \overline{\nu}_k (p_k - \overline{p}_k - \alpha_k^\top \mu + ||\alpha_k||_{\Sigma} \Phi_k^{-1})\\
&+ \lambda\big(D - \sum_k p_k - \sum_i \tilde{W}_i\big) + \sum_i \chi_i \big(1 - \sum_k \alpha_{ki}\big).
\end{align*}
The KKT conditions associated with the stochastic market clearing problem therefore read
\begin{align*}
\frac{\partial \mathcal{L}}{\partial p_k} =& c_k^L + 2 c_k^Q (p_k - \alpha_k^\top \mu) - \underline{\nu}_k + \overline{\nu}_k - \lambda = 0, \forall k,\\
\frac{\partial \mathcal{L}}{\partial \alpha_{ki}} =& - c_k^L \mu_i + 2 c_k^Q\Big(\alpha_{ki} \mu_i^2 - p_k \mu_i + \sigma_i^2 \alpha_{ki} + \sum_{i, j>i} \kappa_{ij} \alpha_{kj}\Big)\\
&+ \underline{\nu}_k \Big(\mu_i + \Phi_k^{-1} \frac{\sigma_i^2 \alpha_{ki} + \sum_{i, j>i} \kappa_{ij} \alpha_{kj}}{||\alpha_k||_{\Sigma}}\Big)\\
&+ \overline{\nu}_k \Big(\mu_i + \Phi_k^{-1} \frac{\sigma_i^2 \alpha_{ki} + \sum_{i, j>i} \kappa_{ij} \alpha_{kj}}{||\alpha_k||_{\Sigma}}\Big) - \chi_i = 0, \forall k, \forall i,\\
\sum_k p_k +& \sum_i \tilde{W}_i = D,\\
\sum_k \alpha_{ki} &= 1, \forall i,\\
0 \le \overline{\nu}_k &\perp \overline{p}_k + \alpha_k^\top \mu - ||\alpha_k||_{\Sigma} \Phi_k^{-1} - p_k \ge 0, \forall k,\\
0 \le \underline{\nu}_k & \perp p_k - \alpha_k^\top \mu - ||\alpha_k||_{\Sigma}\Phi_k^{-1} \ge 0, \forall k,
\end{align*}
from which the range constraints of $\alpha_{ki}$ variables were omitted for conciseness. Since the objective function is convex, the functions forming the constraints are continuously differentiable and the problem satisfies Slater's condition, the KKT conditions are necessary and sufficient. Thus, a solution to the complementarity problem gives a (globally) optimal solution of the market clearing problem. In addition, since the objective is strongly convex and the feasible set is convex (and non-empty), there is also a unique optimal solution to the problem.

\section*{Appendix B}
We derive the Lagrange dual of the following quadratic program:
\begin{align*}
\underset{x}{\min} \hspace{10pt} & \frac{1}{2}x^\top Q x + c^\top x\\
\mbox{s.t. } & Ax \le b \hspace{15pt} (\pi)\\
&x \in \mathbb{R}^n,
\end{align*}
where $Q \succeq 0$ (i.e., it is positive semi-definite) and $A \in \mathbb{R}^{m \times n}$. We start by writing the Lagrange function
\begin{equation*}
\mathcal{L}(x, \pi) = \frac{1}{2}x^\top Q x + c^\top x + \pi^\top(b - Ax)
\end{equation*}
and form the dual function
\begin{equation*}
d(\pi) = \underset{x \in \mathbb{R}^n}{\inf} \mathcal{L}(x, \pi).
\end{equation*}
Stationary points $\hat{x}$ of the Lagrangian must satisfy
\begin{equation*}
Q\hat{x} = A^\top \pi -c,
\end{equation*}
which can simply be obtained by setting the partial derivatives of $\mathcal{L}$ with respect to the entries of $x$ to $0$. These conditions are necessary for optimality and allow us to write the dual problem
\begin{equation*}
\underset{\pi \le 0}{\max} \mbox{ }d(\pi)
\end{equation*}
as
\begin{align*}
\underset{x, \pi}{\max} \hspace{10pt} & -\frac{1}{2}x^T Q x + \pi^\top b\\
\mbox{s.t. } & Qx = A^\top \pi -c\\
&x \in \mathbb{R}^n, \pi \le 0.
\end{align*}
Note that the objective follows from the identities
\begin{equation*}
c^\top x + \pi^\top(b - Ax) = \pi^\top b - (A^\top \pi - c)^\top x =  \pi^\top b - x^\top Q x,
\end{equation*}
for any pair $(x, \pi)$ jointly satisfying the first-order stationarity conditions given above. Note that $Q^\top = Q$ by definition. 

If $Q \succ 0$ (i.e., it is positive definite), the first-order stationarity conditions can be re-written as
\begin{equation*}
\hat{x} = Q^{-1}(A^\top \pi - c),
\end{equation*}
and the dual problem becomes
\begin{align*}
\underset{\pi}{\max} \hspace{10pt} & -\frac{1}{2}(A^\top \pi -c)^\top Q^{-1} (A^\top \pi -c) + \pi^\top b\\
\mbox{s.t. } & \pi \le 0.
\end{align*}
Note that duality results for (convex) quadratic programs can also be obtained via linear programming duality \cite{Dorn1960}.

\bibliographystyle{plain} % We choose the "plain" reference style
\bibliography{references}

\end{document}