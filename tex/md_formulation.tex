\documentclass{article}
\usepackage[utf8]{inputenc}
\usepackage{amsmath}
\usepackage{amsthm}
\usepackage{amssymb}
\usepackage{mathabx}
\usepackage{xcolor}
\newtheorem{theorem}{Theorem}
\newtheorem{proposition}{Proposition}
\newtheorem{definition}{Definition}
\newtheorem{corollary}{Corollary}
\newtheorem{lemma}{Lemma}
\newtheorem{assumption}{Assumption}

\title{Game Theory \& Mechanism Design Primer}
\author{Mathias Berger}
\date{Spring 2022}

\begin{document}

\maketitle

\section{Prelude}

We informally introduce some useful concepts and definitions from the fields of economics, game theory and mechanism design.

\subsection{Economics}

\begin{definition}
(Pareto Optimality) An outcome from which any attempt to benefit a given player by deviating to some other outcome will necessarily result in a loss in utility to someone else.
\end{definition}

In economics, the concept of \textit{efficiency} is typically understood as a Pareto-optimal outcome.

\begin{definition}
(Competitive Equilibrium) An outcome that clears the market and maximises the pay-off of market participants.
\end{definition}

\begin{definition}
(Price-Taking) A situation where producers cannot single-handedly influence market outcomes and prices.
\end{definition}  

\begin{definition}
(Market Power) The ability of a producer to unilaterally influence market outcomes and prices. 
\end{definition}

In the context of electricity markets, a useful term is the following.

\begin{definition}
(Linear Pricing) All electricity traded via an exchange at a given location and time period is traded at a unique market-clearing price.
\end{definition}

\subsection{Game Theory}

In its basic form, a game is as follows.

\begin{definition}
(Game) A situation in which players make rational decisions according to defined rules in an attempt to receive some sort of pay-off.
\end{definition}

\begin{definition}
(Utility) A measure of the preferences of an agent.
\end{definition}

\begin{definition}
(Strategy) Any of the options that a player chooses in a setting where the outcome depends not only on her own actions but on the actions of others.
\end{definition}

\begin{definition}
(One-Shot Simultaneous Move Game) A game in which all players simultaneously choose an action from their set of possible strategies.
\end{definition}

\begin{definition}
(Full-Information Game) A game in which all players know the utilities and strategies of other players.
\end{definition}

\begin{definition}
(Pure Strategy) Deterministically selecting an action and playing it.
\end{definition}

\begin{definition}
(Mixed Strategy) Assigning a probability to each action, randomly selecting an action and playing it.
\end{definition}

In other words, a mixed strategy can be interpreted as a randomised strategy, whereby a player picks a probability distribution over her set of pure strategies and independently selects a strategy at random based on this distribution.

\begin{definition}
(Dominant Strategy) The strategy chosen by a player is dominant if it maximises her utility, irrespective of what other players do.
\end{definition}

\begin{definition}
(Solution Concept) A formal rule for predicting how a game will be played. These predictions are called "solutions", and describe which strategies will be adopted by players and, therefore, the result of the game.
\end{definition}

\begin{definition}
(Non-Cooperative Game) A game in which all players act selfishly and seek to deviate alone from proposed solutions.
\end{definition}

For non-cooperative games, different types of solution concepts can be employed.

\begin{definition}
(Dominant Strategy Equilibrium) An outcome in which every player is doing the best she possibly can irrespective of what other players do.
\end{definition}

\begin{definition}
(Nash Equilibrium) In a game of complete information, a Nash equilibrium is a strategy profile
where each player's strategy is a best response to the strategies of the other players as given
by the strategy profile.
\end{definition}

Put differently, in a Nash Equilibrium, every player is doing the best she possibly can given other players' choices: no player can benefit from unilaterally changing her choice. Note that a Nash equilibrium may be strictly Pareto inefficient. This is notably the case in the Prisoner's dilemma, in the sense that there is an outcome other than the (unique) Nash equilibrium in which all of the players achieve a smaller cost.

\begin{definition}
(Pure-Strategy Nash Equilibrium) A Nash equilibrium in which each player deterministically plays her chosen strategy.
\end{definition}

In some settings, pure-strategy Nash equilibria may not exist, which leads to the introduction of the following concept.

\begin{definition}
(Mixed-Strategy Nash Equilibrium) A Nash equilibrium in which each player plays a mixed-strategy.
\end{definition}

Note that players are typically assumed to be risk-neutral in mixed-strategy Nash equilibria. A weaker notion of equilibrium also exists.

\begin{definition}
(Correlated Equilibrium) An outcome in which no single player can unilaterally improve its expected profit by switching to a different strategy.
\end{definition}

\begin{definition}
(Incomplete-Information Game) A game in which players have limited information about the utilities and strategies of other players.
\end{definition}

Incomplete-information games are often referred to as \textit{Bayesian} games.

\begin{definition}
(Type) The private information held by an agent.
\end{definition}

\begin{definition}
(Independent Private Values) The utility of a player depends fully on his private information and not on any information of others as it is independent from his own information.
\end{definition}

\begin{definition}
(Strategy) A strategy in a game of incomplete information is a function that maps an agent's type to
any of the agent's possible actions in the game.
\end{definition}

The concept of a Nash equilibrium can be also extended to a game of incomplete information.

\begin{definition}
(Bayesian Nash Equilibrium) A strategy profile such that no player can increase their ex-ante expected utility by unilaterally changing their strategy.
\end{definition}

In this context, when computing their ex-ante expected utility, players take the expectation with respect to a distribution over types. This distribution, which is assumed known by all players, is assumed to reflect prior (probabilistic) knowledge about types. If no such probabilistic information about players' types is available (a setting sometimes referred to as \textit{pre-Bayesian}), a different solution concept must be used. 

\begin{definition}
(Ex-Post Nash Equilibrium) A strategy profile such that the strategy of each player is a best response to the strategies of all other players for all possible values of their types.
\end{definition}

\begin{definition}
(Cooperative Game) A game in which groups of players coordinate their actions.
\end{definition}

\begin{definition}
(Coalition) Any subset of players.
\end{definition}

In the context of full-information cooperative games, the Nash equilibrium concept can be extended as follows.

\begin{definition}
(Strong Nash Equilibrium) An outcome in which no subset of players has a way to simultaneously change their strategies in order to improve each of the participant's welfare.
\end{definition}

\subsection{Mechanism Design}

Roughly, mechanism design aims to design games that have dominant strategy solutions and such that these solutions lead to desirable outcomes (e.g., socially desirable). Alternatively, it seeks to implement social choice functions by pairing them with payments that induce players to report their preferences truthfully. In mechanism design and auction theory, an efficient outcome is sometimes defined as an outcome maximising the sum of utilities of all players (so-called social welfare).

\begin{definition}
(Mechanism) A mechanism is defined by a social choice function (which maps players' preferences or valuations to an outcome or allocation) and a set of payment functions (which determine how players are rewarded or penalised).
\end{definition}

\textcolor{red}{clarify distinction between "mechanism" and "revelation mechanism"; does the concept of "indirect revelation mechanism" make sense?; link with revelation principle?}

\begin{definition}
(Revelation Mechanism) \textcolor{red}{TBA}
\end{definition}

\begin{definition}
(Direct Mechanism) A direct mechanism is one in which the space of possible actions is equal to the space of possible types.
\end{definition}

\begin{definition}
(Individual Rationality) The property that the utility of players is always non-negative for outcomes of the mechanism.
\end{definition}

This property is sometimes referred to as \textit{cost recovery}.

\begin{definition}
(Dominant-Strategy Incentive Compatibility) A mechanism is dominant-strategy incentive compatible (DSIC) if it is a dominant strategy for each player to report her private information truthfully.
\end{definition}

\begin{definition}
(Bayesian Incentive Compatibility) A mechanism is Bayesian incentive compatible (BIC) if truthful bidding results in a Bayesian-Nash equilibrium.
\end{definition}

\begin{definition}
(Ex-Post Incentive Compatibility) A mechanism is ex-post incentive compatible (EPIC) if truthful bidding results in an ex-post Nash equilibrium.
\end{definition}

A requirement that all players have non-negative utilities at equilibrium is sometimes added to these definitions.

\begin{definition}
(Implementation) A mechanism implements a social choice function if some equilibrium of the game it induces leads to outcomes that coincide with outcomes produced by said choice function for all possible values of players' types.
\end{definition}

The type of implementation depends on the type of players' strategies and associated equilibrium.

\begin{definition}
(False-Name Bidding) A situation where a player places multiple bids under false names in order to influence the outcome of the auction and increase her utility.
\end{definition}

False-name bidding is sometimes referred to as \textit{shill bidding}.

\begin{definition}
(Collusion) A situation where a coalition of players coordinate their strategies to increase their respective utilities.
\end{definition}

\begin{proposition}
(Revelation Principle) If there exists an arbitrary mechanism that implements a given social choice function in dominant strategies, then there exists an incentive compatible mechanism that implements this same social choice function. The payments of the players in the incentive compatible mechanism are identical to those, obtained at equilibrium, of the original mechanism.
\end{proposition}

\begin{proposition}
(Uniqueness of Prices) Given a social choice function and a set of payment functions defining a (dominant-strategy) incentive compatible mechanism, a new mechanism constructed by taking the same social choice function and changing payment functions is incentive compatible if and only if the payment function of each player in the new mechanism can be expressed as the sum of i) the payment function of the same player in the original mechanism, ii) some function that does not depend on the valuation function of this player.
\end{proposition}

\begin{corollary}
(Social Welfare and Incentive Compatibility) The only (dominant-strategy) incentive compatible mechanisms that maximise social welfare are those with Vickrey-Clarke-Groves payments.
\end{corollary}

\begin{definition}
(Incentive Efficiency) A direct mechanism is said to be incentive efficient if it maximises some weighted sum of the agents' expected pay-offs subject to their incentive compatibility constraints.
\end{definition}

\begin{definition}
(Price of Anarchy) The price of anarchy of a game, for some choice of objective function and equilibrium concept, is defined as the ratio between the worst objective function value of an equilibrium of the game and that of an optimal outcome (i.e., an outcome optimising the chosen objective).
\end{definition}

We would ideally like the price of anarchy to be as close to 1 as possible. Note that in the case of games with multiple equilibria, the price of anarchy will be large even if only one of these equilibria is highly inefficient. 

\begin{definition}
(Price of Stability) The price of stability, for some choice of objective function and equilibrium concept, is a measure of inefficiency designed to differentiate between games in which all equilibria are inefficient and those in which some equilibrium is inefficient. It can be computed as the ratio between the best objective function value of one of its equilibria and that of an optimal outcome (i.e., an outcome optimising the chosen objective).
\end{definition}

Note that the price of anarchy and the price of stability are the same in games with a unique equilibrium.

\section{Formulation}

We consider the case of a wholesale electricity market with inelastic demand and model it as a sealed-bid reverse auction. Let us consider a set $\mathcal{G} = \{1, \ldots, n\}$ of power generators (the bidders) and a market operator (the auctioneer). We consider a set of possible outcomes $\mathcal{D} = \bigtimes_g \mathcal{D}_g$, where $\mathcal{D}_g$ represents the set of possible outcomes of generator $g$. We assume that each generator $g \in \mathcal{G}$ has cost function $c_g(\hat{\beta}_g): \mathcal{D} \rightarrow \mathbb{R}_+$. The cost function $c_g(\beta_g)$ of generator $g$ is assumed to be parametrised by $\beta_g \in \mathcal{B}_g$, where $\mathcal{B}_g$ is the set of possible parameters for generator $g$. The functional form of the cost function of generator $g$ is known to the auctioneer but the true parameter values $\hat{\beta}_g$ are private (i.e., only known to $g$). Note that $\hat{\beta}_g$ may be a vector or scalar parameter, depending on the context, and it usually has non-negative entries. In what follows, we will use $c_g = c_g(\beta_g)$ and $\hat{c}_g = c_g(\hat{\beta}_g)$ as shorthand where appropriate.

The auction process works as follows. First, each generator willing to take part in the auction is required to post a bid $\beta_g \in \mathcal{B}_g$. The collection of bids from all generators forms a bid profile $\beta \in \mathcal{B} = \bigtimes_g \mathcal{B}_g$. Then, a mechanism defines the rules of the auction, which take the form of a (social) choice function and a set of payment functions. More formally, a \textit{choice} function $f: \mathcal{B} \rightarrow \mathcal{D}$ maps bid profiles provided by participants to outcomes, indicating whether (and to what extent) bids are accepted. Note that the choice function is sometimes referred to as the \textit{decision}, \textit{outcome} or \textit{allocation} function in the literature. On the other hand, \textit{payment} functions $p_g: \mathcal{B} \rightarrow \mathbb{R}, \forall g \in \mathcal{G},$  can be interpreted as mapping bid profiles to payments made to each generator. We assume that generators have quasilinear preferences, that is, the utility functions of generators can be expressed as the difference between the monetary payments they receive for their bids and their intrinsic costs for the outcome, $u_g(\beta) = p_g(\beta) - \hat{c}_g\big(f(\beta)\big), \forall g$.

The mechanism induces a game of incomplete information.

\section{Examples}

\subsection{Pay-as-Bid}

\subsection{Locational Marginal Pricing}

\subsection{Vickrey-Clarke-Groves}

\textcolor{red}{Provide general payment rule and then Clarke Pivot rule}

\section{Application to Chance-Constrained Markets}

\section{Comments/Questions}

\begin{itemize}
\item In the chance-constrained setting, the cost of flexible generators will depend upon the variance/covariance of the forecast error. If we assume that the latter is only known by wind producers, flexible generators will need to make assumptions about other players' information to compute their own cost. More specifically, the dependence of the cost function on these parameters would be explicit (through the term involving the covariance matrix) rather than implicit (through the outcome of the allocation process, which may be influenced by other players' strategies). Can this setting be handled in standard game theory or is it generally assumed that all players can compute their own costs based on their own private information (type)? Does some concept of Bayesian-Nash equilibrium suffice to capture this kind of set-up?
\item In the current setting, flexible generators are not explicitly bidding for reserves (i.e., they are not providing explicit reserve bids like they do for power generation). Would it make sense/be useful at all to introduce reserve bids as well?
\item In the definition of the optimisation problem from which the social choice function is derived, the wind production should be turned into a variable. Indeed, in the current formulation, if the load is smaller than the expected wind production, the problem will be infeasible. It is not entirely clear how it should be bounded, however. It would probably make sense to introduce chance constraints on wind generation bounds. 
\item The bid structure of renewable power plants should be clarified: to what product does a "variance bid" correspond? For instance, in the case of flexible generators, a "generation cost/capacity" bid will correspond to some power generation product 
\item In electricity market applications, we don't have much flexibility when picking the choice/allocation function: in most cases, it will be defined as the argmax of some social welfare maximisation formulation including a variety of operational and physical (e.g., power flow) constraints. This means that the only degree of freedom we have when designing electricity market mechanisms are payment functions. 
\item A theoretical result in Roughgarden's book (Chapter 9, one of the corollaries to Theorem 9.37) states that the VCG payment rule is the only one yielding an incentive-compatible mechanism when the social choice function is social welfare maximisation. I guess that this result was derived for basic social welfare maximisation without any constraints like the ones typically found in electricity market applications. To what extent is that result applicable to electricity markets? My guess is that if it does not work for the unconstrained case, it is highly unlikely to work in the constrained case (since the former can be seen as a relaxation of the 
\item What would an application of the VCG mechanism look like for a variance product?
\item Most of the literature on incentive compatibility in mechanism design makes use of the concept of dominant-strategy incentive-compatibility (DSIC). Would it make sense to relax these requirements and work with different notions of incentive compatibility (e.g., Bayesian-Nash or ex-Post Nash incentive compatibility)?
\end{itemize}

\bibliographystyle{plain} % We choose the "plain" reference style
\bibliography{references}

\end{document}