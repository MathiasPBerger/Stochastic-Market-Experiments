\documentclass{article}
\usepackage[utf8]{inputenc}
\usepackage{amsmath}
\usepackage{amsthm}
\usepackage{amssymb}
\usepackage{mathabx}
\usepackage{xcolor}
\newtheorem{theorem}{Theorem}
\newtheorem{proposition}{Proposition}
\newtheorem{definition}{Definition}
\newtheorem{corollary}{Corollary}
\newtheorem{lemma}{Lemma}
\newtheorem{assumption}{Assumption}

\title{Game Theory \& Mechanism Design Primer}
\author{Mathias Berger}
\date{Spring 2022}

\begin{document}

\maketitle

\section{Prelude}

We informally introduce some useful concepts and definitions from the fields of economics, game theory and mechanism design.

\subsection{Economics}

\begin{definition}
(Pareto Optimality) An outcome from which any attempt to benefit a given player by deviating to some other outcome will necessarily result in a loss in utility to someone else.
\end{definition}

In economics, the concept of \textit{efficiency} is typically understood as a Pareto-optimal outcome.

\begin{definition}
(Competitive Equilibrium) An outcome that clears the market and maximises the pay-off of market participants.
\end{definition}

\begin{definition}
(Price-Taking) A situation where producers cannot single-handedly influence market outcomes and prices.
\end{definition}  

\begin{definition}
(Market Power) The ability of a producer to unilaterally influence market outcomes and prices. 
\end{definition}

In the context of electricity markets, a useful term is the following.

\begin{definition}
(Linear Pricing) All electricity traded via an exchange at a given location and time period is traded at a unique market-clearing price.
\end{definition}

\subsection{Game Theory}

In its basic form, a game is as follows.

\begin{definition}
(Game) A situation in which players make rational decisions according to defined rules in an attempt to receive some sort of pay-off.
\end{definition}

\begin{definition}
(Utility) A measure of the preferences of an agent.
\end{definition}

The utility of an agent is often referred to as its \textit{pay-off} in the literature.

\begin{definition}
(Strategy) Any of the options that a player chooses in a setting where the outcome depends not only on her own actions but on the actions of others.
\end{definition}

\begin{definition}
(One-Shot Simultaneous Move Game) A game in which all players simultaneously choose an action from their set of possible strategies.
\end{definition}

\begin{definition}
(Full-Information Game) A game in which all players know the utilities and strategies of other players.
\end{definition}

\begin{definition}
(Pure Strategy) Deterministically selecting an action and playing it.
\end{definition}

\begin{definition}
(Mixed Strategy) Assigning a probability to each action, randomly selecting an action and playing it.
\end{definition}

In other words, a mixed strategy can be interpreted as a randomised strategy, whereby a player picks a probability distribution over her set of pure strategies and independently selects a strategy at random based on this distribution.

\begin{definition}
(Dominant Strategy) The strategy chosen by a player is dominant if it maximises her utility, irrespective of what other players do.
\end{definition}

\begin{definition}
(Solution Concept) A formal rule for predicting how a game will be played. These predictions are called "solutions", and describe which strategies will be adopted by players and, therefore, the result of the game.
\end{definition}

\begin{definition}
(Non-Cooperative Game) A game in which all players act selfishly and seek to deviate alone from proposed solutions.
\end{definition}

For non-cooperative games, different types of solution concepts can be employed.

\begin{definition}
(Dominant Strategy Equilibrium) An outcome in which every player is doing the best she possibly can irrespective of what other players do.
\end{definition}

\begin{definition}
(Nash Equilibrium) In a game of complete information, a Nash equilibrium is a strategy profile
where each player's strategy is a best response to the strategies of the other players as given
by the strategy profile.
\end{definition}

Put differently, in a Nash Equilibrium, every player is doing the best she possibly can given other players' choices: no player can benefit from unilaterally changing her choice. Note that a Nash equilibrium may be strictly Pareto inefficient. This is notably the case in the Prisoner's dilemma, in the sense that there is an outcome other than the (unique) Nash equilibrium in which all of the players achieve a smaller cost.

\begin{definition}
(Pure-Strategy Nash Equilibrium) A Nash equilibrium in which each player deterministically plays her chosen strategy.
\end{definition}

In some settings, pure-strategy Nash equilibria may not exist, which leads to the introduction of the following concept.

\begin{definition}
(Mixed-Strategy Nash Equilibrium) A Nash equilibrium in which each player plays a mixed-strategy.
\end{definition}

Note that players are typically assumed to be risk-neutral in mixed-strategy Nash equilibria. A weaker notion of equilibrium also exists.

\begin{definition}
(Correlated Equilibrium) An outcome in which no single player can unilaterally improve its expected profit by switching to a different strategy.
\end{definition}

\begin{definition}
(Incomplete-Information Game) A game in which players have limited information about the utilities and strategies of other players.
\end{definition}

Incomplete-information games are often referred to as \textit{Bayesian} games.

\begin{definition}
(Type) The private information held by an agent.
\end{definition}

Note that types may be either independent or correlated. 

\begin{definition}
(Independent Private Values) The valuation of a player depends fully on his private information and not on any information of others as it is independent from his own information.
\end{definition}

This setting is sometimes simply referred to as a \textit{private values environment} in the literature.

\begin{definition}
(Interdependent Values) The valuation of a player depends not only on his private information but also (explicitly) on that of other players.
\end{definition}

This is sometimes referred to as the \textit{common values} setting in the literature. Alternatively, the concept of \textit{informational externalities} is sometimes invoked to describe a setting with interdependent valuations. Note that interdependent values may arise in a context where agents have either independent types or correlated types.

\begin{definition}
(Strategy) A strategy in a game of incomplete information is a function that maps an agent's type to
any of the agent's possible actions in the game.
\end{definition}

The concept of a Nash equilibrium can be also extended to a game of incomplete information.

\begin{definition}
(Bayesian Nash Equilibrium) A strategy profile such that no player can increase their ex-ante expected utility by unilaterally changing their strategy.
\end{definition}

In this context, when computing their ex-ante expected utility, players take the expectation with respect to a distribution over types. This distribution, which is assumed known by all players, is assumed to reflect prior (probabilistic) knowledge about types. If no such probabilistic information about players' types is available (a setting sometimes referred to as \textit{pre-Bayesian}), a different solution concept must be used. 

%\begin{definition}
%(Ex-Ante Nash Equilibrium) \textcolor{red}{TBA}
%\end{definition}

\begin{definition}
(Ex-Interim Nash Equilibrium) \textcolor{red}{TBA}
\end{definition}

\begin{definition}
(Ex-Post Nash Equilibrium) A strategy profile such that the strategy of each player is a best response to the strategies of all other players for all possible values of their types.
\end{definition}

\begin{definition}
(Cooperative Game) A game in which groups of players coordinate their actions.
\end{definition}

\begin{definition}
(Coalition) Any subset of players.
\end{definition}

In the context of full-information cooperative games, the Nash equilibrium concept can be extended as follows.

\begin{definition}
(Strong Nash Equilibrium) An outcome in which no subset of players has a way to simultaneously change their strategies in order to improve each of the participant's welfare.
\end{definition}

\subsection{Mechanism Design}

Roughly, mechanism design aims to design games that have dominant strategy solutions and such that these solutions lead to desirable outcomes (e.g., socially desirable). Alternatively, it seeks to implement social choice functions by pairing an outcome function with payments that induce players to report their preferences truthfully. In mechanism design and auction theory, an efficient outcome is sometimes defined as an outcome maximising the sum of utilities of all players (so-called social welfare).

\begin{definition}
(Mechanism) A mechanism is defined by a social choice function (which maps players' preferences or valuations to an outcome or allocation) and a set of payment functions (which determine how players are rewarded or penalised).
\end{definition}


\begin{definition}
(Direct Mechanism) A direct mechanism is one in which the space of possible actions is equal to the space of possible types.
\end{definition}

A direct mechanism is sometimes referred to as a \textit{direct revelation mechanism} in the literature.

\begin{definition}
(Individual Rationality) The property that the utility of players is always non-negative for outcomes of the mechanism.
\end{definition}

This property is sometimes referred to as \textit{cost recovery}.

\begin{definition}
(Dominant-Strategy Incentive Compatibility) A mechanism is dominant-strategy incentive compatible (DSIC) if it is a dominant strategy for each player to report her private information truthfully.
\end{definition}

\begin{definition}
(Bayesian Incentive Compatibility) A mechanism is Bayesian incentive compatible (BIC) if truthful bidding results in a Bayesian-Nash equilibrium.
\end{definition}

\begin{definition}
(Ex-Post Incentive Compatibility) A mechanism is ex-post incentive compatible (EPIC) if truthful bidding results in an ex-post Nash equilibrium.
\end{definition}

A requirement that all players have non-negative utilities at equilibrium is sometimes added to these definitions.

\begin{definition}
(Implementation) A mechanism implements a social choice function if some equilibrium of the game it induces leads to outcomes that coincide with outcomes produced by said choice function for all possible values of players' types.
\end{definition}

The type of implementation depends on the type of players' strategies and associated equilibrium.

\begin{definition}
(False-Name Bidding) A situation where a player places multiple bids under false names in order to influence the outcome of the auction and increase her utility.
\end{definition}

False-name bidding is sometimes referred to as \textit{shill bidding}.

\begin{definition}
(Collusion) A situation where a coalition of players coordinate their strategies to increase their respective utilities.
\end{definition}

\begin{proposition}
(Revelation Principle) If there exists an arbitrary mechanism that implements a given social choice function in dominant strategies, then there exists an incentive compatible mechanism that implements this same social choice function. The payments of the players in the incentive compatible mechanism are identical to those, obtained at equilibrium, of the original mechanism.
\end{proposition}

\begin{proposition}
(Uniqueness of Prices) Given a social choice function and a set of payment functions defining a (dominant-strategy) incentive compatible mechanism, a new mechanism constructed by taking the same social choice function and changing payment functions is incentive compatible if and only if the payment function of each player in the new mechanism can be expressed as the sum of i) the payment function of the same player in the original mechanism, ii) some function that does not depend on the valuation function of this player.
\end{proposition}

\begin{corollary}
(Social Welfare and Incentive Compatibility) The only (dominant-strategy) incentive compatible mechanisms that maximise social welfare are those with Vickrey-Clarke-Groves payments.
\end{corollary}

\begin{definition}
(Incentive Efficiency) A direct mechanism is said to be incentive efficient if it maximises some weighted sum of the agents' expected pay-offs subject to their incentive compatibility constraints.
\end{definition}

\begin{definition}
(Price of Anarchy) The price of anarchy of a game, for some choice of objective function and equilibrium concept, is defined as the ratio between the worst objective function value of an equilibrium of the game and that of an optimal outcome (i.e., an outcome optimising the chosen objective).
\end{definition}

We would ideally like the price of anarchy to be as close to 1 as possible. Note that in the case of games with multiple equilibria, the price of anarchy will be large even if only one of these equilibria is highly inefficient. 

\begin{definition}
(Price of Stability) The price of stability, for some choice of objective function and equilibrium concept, is a measure of inefficiency designed to differentiate between games in which all equilibria are inefficient and those in which some equilibrium is inefficient. It can be computed as the ratio between the best objective function value of one of its equilibria and that of an optimal outcome (i.e., an outcome optimising the chosen objective).
\end{definition}

Note that the price of anarchy and the price of stability are the same in games with a unique equilibrium.

\section{Formulation}

We consider the case of a wholesale electricity market with inelastic demand and model it as a sealed-bid reverse auction. Let us consider a set $\mathcal{G} = \{1, \ldots, n\}$ of power generators (the bidders) and a market operator (the auctioneer). We consider a set of possible outcomes $\mathcal{D} = \bigtimes_g \mathcal{D}_g$, where $\mathcal{D}_g$ represents the set of possible outcomes of generator $g$. We assume that each generator $g \in \mathcal{G}$ has cost function $c_g(\hat{\beta}_g): \mathcal{D} \rightarrow \mathbb{R}_+$. The cost function $c_g(\beta_g)$ of generator $g$ is assumed to be parametrised by $\beta_g \in \mathcal{B}_g$, where $\mathcal{B}_g$ is the set of possible parameters for generator $g$. The functional form of the cost function of generator $g$ is known to the auctioneer but the true parameter values $\hat{\beta}_g$ are private (i.e., only known to $g$). Note that $\hat{\beta}_g$ may be a vector or scalar parameter, depending on the context, and it usually has non-negative entries. In what follows, we will use $c_g = c_g(\beta_g)$ and $\hat{c}_g = c_g(\hat{\beta}_g)$ as shorthand where appropriate.

The auction process works as follows. First, each generator willing to take part in the auction is required to post a bid $\beta_g \in \mathcal{B}_g$. The collection of bids from all generators forms a bid profile $\beta \in \mathcal{B} = \bigtimes_g \mathcal{B}_g$. Then, a mechanism defines the rules of the auction, which take the form of a (social) choice function and a set of payment functions. More formally, a \textit{choice} function $\mathfrak{f}: \mathcal{B} \rightarrow \mathcal{D}$ maps bid profiles provided by participants to outcomes, indicating whether (and to what extent) bids are accepted. Note that the choice function is sometimes referred to as the \textit{decision}, \textit{outcome} or \textit{allocation} function in the literature. On the other hand, \textit{payment} functions $\mathfrak{p}_g: \mathcal{B} \rightarrow \mathbb{R}, \forall g \in \mathcal{G},$  can be interpreted as mapping bid profiles to payments made to each generator. We assume that generators have quasilinear preferences, that is, the utility functions of generators can be expressed as the difference between the monetary payments they receive for their bids and their intrinsic costs for the outcome, $u_g(\beta) = \mathfrak{p}_g(\beta) - \hat{c}_g\big(\mathfrak{f}(\beta)\big), \forall g$.

The mechanism induces a game of incomplete information.

\section{Deterministic Energy-Only Market}

We consider an energy-only deterministic market design featuring a set of flexible generators and cast the problem in the mechanism design framework introduced above.

\subsection{Cost Functions \& Epistemic Types}

We assume that flexible generator $g$ has a quadratic generation cost function, which can be expressed as $c_g(p_g) = c_g^L p_g + c_g^Q p_g^2$ for some generation level $0 \le p_g \le \overline{p}_g$. Production cost coefficients $c_g^L$ and $c_g^Q$ as well as the maximum power generation level $\overline{p}_g$ are gathered into a bid $\beta_g = \{c_g^L, c_g^Q, \overline{p}_g\} \in \mathcal{B}_g$. The true values of these parameters are assumed to be private and define the type of flexible generator $g$. 

\subsection{Choice Function}

The choice function $\mathfrak{f}$ maps a bid profile $\beta \in \mathcal{B}$ to dispatch decisions $p_g \in \mathcal{D}_g$. For simplicity, we write $p_g = \mathfrak{f}_g(\beta)$. The choice function is defined through the solution of an economic dispatch problem, that is,
\begin{align}
\mathfrak{f}(\beta) \in \arg \underset{\{p_g\}_{\forall g}}{\min} \hspace{10pt} & \sum_g \Big(c_g^Lp_g + c_g^Q p_g^2 \Big)\\
\mbox{s.t. } & p_g \le \overline{p}_g,\forall g,\\
& 0 \le p_g,\forall g,\\
& \sum_g p_g + \sum_i \tilde{W}_i = D, \mbox{ } (\lambda)\\
& p_g \in \mathbb{R}, \forall g.
\end{align}

\subsection{Payment Functions}

We consider three different payment rules, namely \textit{pay-as-bid}, \textit{(locational) marginal pricing} and \textit{Vickrey-Clarke-Groves}.

\subsubsection{Pay-as-Bid}

Under the pay-as-bid scheme, generator $g$ is paid the amount that it bid, that is,
\begin{equation*}
\mathfrak{p}_g(\beta) = c_g^L \mathfrak{f}_g(\beta) + c_g^Q \mathfrak{f}_g(\beta)^2, \forall g.
\end{equation*}

\subsubsection{Locational Marginal Pricing}

Under locational marginal pricing, the dual variable $\lambda$ of the energy balance equation is used to compute payments,
\begin{equation*}
\mathfrak{p}_g(\beta) = \lambda \mathfrak{f}_g(\beta), \forall g.
\end{equation*}
Note that $\lambda$ also depends upon $\beta$, although this dependence is implicit.

\subsubsection{Vickrey-Clarke-Groves}

Vickrey-Clarke-Groves (VCG) payments have the general form
\begin{align*}
\mathfrak{p}_g(\beta) &= c_g\big(\mathfrak{f}_g(\beta)\big) + \Big(h_g(\beta_{-g}) - \sum_{g} c_g\big(\mathfrak{f}_g(\beta)\big) \Big)\\
&= h_g(\beta_{-g}) - \sum_{k \ne g} c_k\big(\mathfrak{f}_k(\beta)\big),
\end{align*}
where $h_g$ is a function that does not depend on $\beta_g$ and $\beta_{-g}$ denotes the set of all bids except $\beta_g$. The most frequent choice of $h_g$ is the so-called \textit{Clarke pivot} rule, which is defined as
\begin{align}
h_g(\beta_{-g}) = \underset{\{p_k\}_{\forall k \ne g}}{\min} \hspace{10pt} & \sum_{k \ne g} \Big(c_k^Lp_k + c_k^Q p_k^2 \Big)\\
\mbox{s.t. } & p_k \le \overline{p}_k,\forall k \ne g,\\
& 0 \le p_k,\forall k \ne g,\\
& \sum_{k \ne g} p_k + \sum_i \tilde{W}_i = D,\\
& p_k \in \mathbb{R}, \forall k \ne g.
\end{align}

Note that this payment rule is well-defined if and only if the economic dispatch problem remains feasible if any generator is taken out of the system (which is usually the case in practice).

\subsection{Utilities}

The utility of generator $g$ is defined based on its cost function and the payment it receives, that is,
\begin{equation}
u_g(\beta) = \mathfrak{p}_g(\beta) - \hat{c}_g\big(\frak{f}_g(\beta)\big), \forall g.
\end{equation}

\section{Stochastic Market for Energy and Reserves}

%\textcolor{red}{show that the mechanism is outcome-efficient by construction (clearly introduce utilities of all agents, including system operator and write their sum); clearly state and show that we are in a setting with interdependent valuations; derive implications for incentive compatibility (especially of traditional Vickrey-Clarke-Groves mechanism)}

We consider the problem of designing a stochastic market co-optimising power generation and reserves. In particular, we seek to devise stochastic market mechanisms possessing some desirable properties such as efficiency (maximising social welfare), individual rationality (cost recovery), budget balance (revenue adequacy) and incentive compatibility (truthful bidding leads to some equilibrium outcome).

\subsection{Mechanism Design Formulation}

The stochastic market design problem is cast as a simultaneous, sealed-bid reverse auction and formulated in the framework of mechanism design. 

\subsubsection{Informal Description}

A set of agents is assumed to be endowed with private information (their \textit{type}) and utility functions. These utility functions are assumed to be quasi-linear in money, that is, they can be expressed as the difference between money transfers and valuation functions measuring agents' preferences for certain market outcomes. Note that valuation functions may explicitly depend on both the agents' own type and the types of other agents, which corresponds to the \textit{interdependent valuations} setting.

We consider direct revelation mechanisms, wherein agents are requested to report their types directly to the mechanism, and we seek to implement a social choice function mapping agents' types to market outcomes (e.g., guaranteeing that the agent with the lowest cost is dispatched). The goal is then to design transfer functions as well as an allocation rule together inducing a game of incomplete information whose equilibria satisfy certain properties. Specifically, the mechanism is said to implement a social choice function if, for all possible agents' types, the outcomes corresponding to some equilibrium of the game the mechanism induces (as given by the allocation rule) coincide with the outcomes prescribed by the social choice function.

\subsubsection{Agents}

We consider two types of agents, namely flexible generators and stochastic wind producers.

\subsubsection{Goods}

Agents can offer two different types of goods in the auction, namely electricity and reserves.

\paragraph{Flexible Generators} Flexible generator $g$ can provide both types of goods. Thus, for generator $g$, market outcomes amount to a pair of power generation $p_g$ and reserve provision $\alpha_g$ levels. Let $d_g = \begin{pmatrix} p_g, \alpha_g \end{pmatrix}$ denote this market outcome and let $\mathcal{D}_g$ be the set of feasible outcomes for generator $g$. We also define $d_G = \bigtimes_g d_g$.

\paragraph{Wind Producers} Wind producer $i$ can only offer electricity in the auction. Hence, for wind producer $i$, market outcomes boil down to a power generation level $d_i = \begin{pmatrix} p_i^w \end{pmatrix}$. Let $\mathcal{D}_i$ be the set of feasible outcomes for wind producer $i$. We also define $d_I = \bigtimes_i d_i$.

\paragraph{} The full market outcome is written as $d = d_G \times d_I$.

\subsubsection{Types}

The functional forms of agents' valuations are assumed publicly known and only their parameters are private.

\paragraph{Flexible Generators} The type of flexible generator $g$ gathers (linear and quadratic) production cost coefficients $c_g^L$ and $c_g^Q$ as well as the maximum power generation level $\overline{p}_g$. Let $\beta_g = \begin{pmatrix} c_g^L, c_g^Q, \overline{p}_g\end{pmatrix}$ denote the type of generator $g$ and let $\mathcal{B}_g = \mathbb{R}_+^3$ be its type space. We also define $\beta_G = \bigtimes_g \beta_g$ and $\mathcal{B}_G = \bigtimes_g \mathcal{B}_g$.

\paragraph{Wind Producers} Let us assume that wind generation forecast errors are statistically independent and have zero mean, and let us also assume that wind producers have a small (linear) production cost coefficient. Then, the type of wind producer $i$ is defined by the standard deviation $\sigma_i$ of its forecast error and its cost coefficient $c_i^w$. Let $\beta_i = \begin{pmatrix}\sigma_i, c_i^w\end{pmatrix}$ denote the type of wind producer $i$ and let $\mathcal{B}_i = \mathbb{R}_+^2$ be its type space. We also define $\beta_I = \bigtimes_i \beta_i$ and $\mathcal{B}_I = \bigtimes_i \mathcal{B}_i$.

\paragraph{} Then, the types of all agents and the full type space are written as $\beta = \begin{pmatrix}\beta_G, \beta_I\end{pmatrix}$ and $\mathcal{B} = \mathcal{B}_G \times \mathcal{B}_I$, respectively. Note that the types of different agents are independent from one another.

\subsubsection{Valuations}

In the mechanism design literature, valuation functions are traditionally used to represent agents' preferences for certain auction outcomes (given information about types). In the context of electricity markets, agents incur costs for supplying electricity and other goods (e.g., reserves), which we count as negative valuations. Hence, the valuation function $v_k: \mathcal{D} \times \mathcal{B} \rightarrow \mathbb{R}$ of agent $k$ is expressed in terms of a cost function $c_k: \mathcal{D} \times \mathcal{B} \rightarrow \mathbb{R}$, such that $v_k(d, \beta) = -c_k(d, \beta)$. In the following, we mostly deal with cost functions.  

\paragraph{Flexible Generators} We assume that flexible generator $g$ has a quadratic cost function $c_g: \mathcal{D}_g \times \mathcal{B}_g \times \mathcal{B}_I \rightarrow \mathbb{R}$ reflecting the expected cost of producing electricity and offering reserve services, given some information about agents' types. This cost function can be expressed as
\begin{equation*}
c_g(d_g, \beta_g, \beta_I) = c_g^L p_g + c_g^Q p_g^2 + c_g^Q \alpha_g^\top \Sigma \alpha_g,
\end{equation*}
where $\Sigma_{ii} = \sigma_i^2$ and $\Sigma_{ij} = 0, i \ne j$. Note that the cost function of generator $g$ explicitly depends on some private information held by all wind producers.

\paragraph{Wind Producers} We assume that wind producer $i$ has a linear cost function $c_i: \mathcal{D}_i \times \mathcal{B}_i \rightarrow \mathbb{R}$ reflecting the expected cost of producing electricity,
\begin{equation*}
c_i(d_i, \beta_i) = c_i^w p_i^w.
\end{equation*}
The linear cost coefficient $c_i^w$ often has a very small value and is frequently set to zero in the power systems literature. Note that the cost function of wind producer $i$ is independent of information held by other agents.

\subsubsection{Allocation Rule}

The allocation rule $\mathfrak{f}: \mathcal{B} \rightarrow \mathcal{D}$ of the mechanism defines the set of goods effectively sold in the auction based on bids placed by agents. Since we consider a direct revelation mechanism, agents are requested to report their types directly to the mechanism, resulting in a bid profile $\beta$ (that may not be truthful) and leading to outcomes $d = \mathfrak{f}(\beta)$. For notational convenience, we also denote by $\mathfrak{f}_k(\beta)$ the outcome(s) for agent $k$. In what follows, $\mathfrak{f}$ will also be referred to as the \textit{choice function}.

In electricity markets, the allocation rule $\mathfrak{f}$ is typically defined via the solution of some kind of economic dispatch problem. In this context, we define $\mathfrak{f}$ via the deterministic-equivalent of a chance-constrained economic dispatch problem,
\begin{align}
\mathfrak{f}(\beta) \in \arg \underset{p_g, \alpha_g,p_i^w}{\min} \hspace{10pt} & \sum_g \Big(c_g^L p_g + c_g^Q \big(p_g^2 + \alpha_g^\top \Sigma \alpha_g\big)\Big) + \sum_i c_i^w p_i^w \\
\mbox{s.t. } & p_g \le \overline{p}_g - ||\alpha_g||_{\Sigma} \phi_g,\forall g,\\
& ||\alpha_g||_{\Sigma} \phi_g \le p_g,\forall g,\\
& p_i^w \le \bar{p}_i^w, \forall i,\\
& \sum_g p_g + \sum_i p_i^w = D, \hspace{30pt} (\lambda)\\
& \sum_g \alpha_{gi} = 1, \forall i, \hspace{50pt} (\chi_i)\\
& 0 \le \alpha_{gi} \le 1, \forall g, \forall i,\\
& \alpha_{gi} \in \mathbb{R}, \forall g, \forall i,\\
& p_g \in \mathbb{R}, \forall g,\\
& p_i^w \in \mathbb{R}, \forall i.
\end{align}

A choice function is said to maximise social welfare if it maximises the sum of valuations of all agents over the set of outcomes. This is notably the case of the allocation rule $\mathfrak{f}$ defined above, as 
\begin{equation*}
\mathfrak{f}(\beta) \in \arg \min_d \bigg[\sum_k c_k(d, \beta)\bigg] = \arg \max_d \bigg[\sum_k v_k(d, \beta) \bigg].
\end{equation*}

It is also worth noting that since the economic dispatch problem above is convex and satisfies standard regularity conditions (e.g., Slater's condition is satisfied), strong duality holds and dual variables $\lambda$ and $\chi_i, \forall i,$ have a direct economic interpretation. They can thus be retrieved as by-products of the economic dispatch problem and used to construct payment rules, as discussed below.

\subsubsection{Transfer Functions}

Transfer functions $\mathfrak{p}: \mathcal{B} \rightarrow \mathbb{R}$ determine money transfers between agents based on their collective bids. In the context of a reverse auction, transfer functions can be interpreted as payment rules rewarding agents for their bids (if they are accepted). We consider two different payment rules, namely some \textit{marginal pricing}-inspired and \textit{Vickrey-Clarke-Groves} (VCG) payments.

\paragraph{Marginal Pricing} This payment rule is inspired by the marginal pricing scheme typically used in day-ahead electricity markets and relies on dual variables $\lambda$ and $\chi_i, \forall i$. The former is used to price electricity, while the latter is used to price reserves. Note that, implicitly, these prices depend on the bid profile $\beta$, that is, $\lambda = \lambda(\beta)$ and $\chi_i = \chi_i(\beta), \forall i$.

Flexible generator $g$ receives money based on the following transfer function,
\begin{equation*}
\mathfrak{p}_g(\beta) = p_g \lambda + \alpha_g^\top \chi,
\end{equation*}
where $\begin{pmatrix} p_g, \alpha_g \end{pmatrix} = \mathfrak{f}_g(\beta)$. Intuitively, this rule rewards flexible generators for supplying electricity and reserves.

Wind producer $i$, on the other hand, receives money based on the following payment rule,
\begin{equation*}
\mathfrak{p}_i(\beta) = p_i^w \lambda - \chi_i.
\end{equation*}
The rationale behind this rule is to reward wind producers for supplying electricity but transfer the cost of providing reserves to them.

\paragraph{Vickrey-Clarke-Groves}

For some agent $k$, Vickrey-Clarke-Groves payments have the general form
\begin{align*}
\mathfrak{p}_k(\beta) &= c_k\big(\mathfrak{f}_k(\beta), \beta \big) + \Big(h_k(\beta_{-k}) - \sum_{k} c_k\big(\mathfrak{f}_k(\beta), \beta \big) \Big)\\
&= h_k(\beta_{-k}) - \sum_{j \ne k} c_j\big(\mathfrak{f}_j(\beta), \beta\big),
\end{align*}
where $h_k: \mathcal{B}_{-k} \rightarrow \mathbb{R}$ is a function that does not depend on $\beta_k$ and $\beta_{-k}$ denotes the set of all bids except $\beta_k$. The most frequent choice of $h_k$ is the so-called \textit{Clarke pivot} rule, which returns the cost of the economic dispatch solution when $k$ is taken out of the system.

In the case of flexible generators, the Clarke pivot can be computed via
\begin{align*}
h_g(\beta_{-g}) = \underset{p_{-g}, \alpha_{-g}, p_i^w}{\min} \hspace{10pt} & \sum_{k \ne g} \Big(c_k^L p_k+ c_k^Q \big(p_k ^2 + \alpha_k^\top \Sigma \alpha_k\big)\Big) + \sum_i c_i^w p_i^w \\
\mbox{s.t. } & p_k \le \overline{p}_k - ||\alpha_k||_{\Sigma} \phi_k,\forall k \ne g,\\
& ||\alpha_k||_{\Sigma} \phi_k \le p_k,\forall k \ne g,\\
& \sum_{k \ne g} p_k + \sum_i p_i^w = D, \hspace{60pt} (\lambda)\\
& \sum_{k \ne g} \alpha_{ki} = 1, \forall i, \hspace{85pt} (\chi_i)\\
& 0 \le \alpha_{ki} \le 1, \forall k \ne g, \forall i,\\
& \alpha_{gi} \in \mathbb{R}, \forall g, \forall i,\\
& p_k \in \mathbb{R}, \forall k \ne g,\\
& p_i^w \in \mathbb{R}, \forall i.
\end{align*}

Recall that the Clarke pivot seeks to evaluate the value of the economic dispatch solution computed when a given agent does not partake in the auction. In the case of wind producer $i$, this implies solving the economic dispatch problem without it. Specifically, this implies removing all parameters and variables associated with it (including reserves from flexible generators meant to balance its variability). Let $\Sigma_{-i}$ and $\alpha_{g, -i}, \forall g,$ denote the matrix and vectors obtained by striking out entries associated with $i$. Then, the Clarke pivot can be obtained as
\begin{align*}
h_i(\beta_{-i}) = \underset{p_g, \alpha_{g,-i}, p_{-i}^w}{\min} \hspace{10pt} & \sum_g \Big(c_g^L p_g + c_g^Q \big(p_g^2 + \alpha_{g,-i}^\top \Sigma_{-i} \alpha_{g,-i}\big)\Big) + \sum_{j \ne i} c_j^w p_j^w \\
\mbox{s.t. } & p_g \le \overline{p}_g - ||\alpha_{g,-i}||_{\Sigma_{-i}} \phi_g,\forall g,\\
& ||\alpha_{g,-i}||_{\Sigma_{-i}} \phi_g \le p_g,\forall g,\\
& p_j^w \le \bar{p}_j^w, \forall j \ne i,\\
& \sum_g p_g + \sum_{j \ne i} p_j^w = D,\\
& \sum_g \alpha_{gj} = 1, \forall j \ne i,\\
& 0 \le \alpha_{gj} \le 1, \forall g, \forall j \ne i,\\
& \alpha_{gi} \in \mathbb{R}, \forall g, \forall i,\\
& p_g \in \mathbb{R}, \forall g,\\
& p_j^w \in \mathbb{R}, \forall j \ne i.
\end{align*}

\subsubsection{Utilities}

\paragraph{Flexible Generators} The utility function $u_g: \mathcal{B} \rightarrow \mathbb{R}$ of flexible generator $g$ measures its net pay-off, based on the bids of all agents and market outcomes. Since its utility is quasi-linear in money, it can be expressed as
\begin{equation*}
u_g(\beta) = \mathfrak{p}_g(\beta) - c_g(\mathfrak{f}_g(\beta), \beta_g, \beta_I).
\end{equation*}

\paragraph{Wind Producers} Likewise, wind producer $i$ has a utility function $u_i: \mathcal{B} \rightarrow \mathbb{R}$ such that $u_i(\beta) = \mathfrak{p}_i(\beta) - c_i(\mathfrak{f}_i(\beta), \beta_i)$.

%\subsubsection{Induced Game}

\subsection{Analysis}

Based on the formulation above, we showed experimentally that the VCG mechanism is \textbf{not} incentive compatible in this setting: untruthful bidding by wind producers is profitable. This essentially results from the fact the their bids also influence the cost functions of flexible generators (the so-called \textit{interdependent valuations} setting). More generally, it is unclear that designing a truthful mechanism maximizing social welfare exactly (without any guarantees on revenue adequacy and cost recovery) is possible at all in our setting. This mostly stems from the fact that we are working in a setting with multi-dimensional, interdependent valuations (i.e., cost functions). A number of key results from the literature on the topic are listed below, but I still need to do some digging.

\begin{itemize}
\item (Positive Result) The VCG mechanism is dominant-strategy incentive-compatible for auctions with multiple goods in the (independent) private values setting \cite{Vickrey1961,Clarke1971,Groves1973}.
\item (Negative Result) In the independent private values setting, the only dominant-strategy incentive-compatible mechanisms that maximise social welfare are those with VCG payments \cite{GreenLaffont1977}.
\item (Negative Result) The only deterministic social choice functions that are ex-post implementable in generic mechanism design frameworks with multi-dimensional signals, interdependent valuations and transferable utilities, are constant functions \cite{Jehiel2006} (i.e., always yielding the same outcome, irrespective of agents' type reports).
\item (Negative Result) When at least one agent's signal is two-dimensional (and the distribution of signals is independent across agents), Jehiel and Moldovanu \cite{Jehiel2001} have shown that, for generic valuations, the efficient social choice rule is not Bayesian implementable, and hence \textit{a fortiori} not ex-post implementable.
\item (Positive Result) Efficient ex-post implementation is possible for interdependent valuations when signals (types) are one-dimensional and satisfy a so-called single-crossing property \cite{Dasgupta2000}.
\item (Positive Result) In a setting with interdependent valuations and multi-dimensional correlated types, McLean and Postlewaite \cite{McLeanPostlewaite2004} obtain approximate efficiency in a Bayes-Nash equilibrium. Their results rely on the concept of \textit{informational size}, which measures the amount of information the type of an agent provides about the unobserved object on which agents' types depend, given the types of other agents.
\item \textcolor{orange}{Our valuations are separable in the sense of Jehiel et al. \cite{Jehiel2006}. There may therefore be some hope of designing an ex-post incentive-compatible mechanism based on some affine-maximization properties of implementable social choice functions. Look into Jehiel et al. \cite{Jehiel2008} and Lavi et al. \cite{Lavi2003}.}
\end{itemize}


\section{Comments/Questions}

\begin{itemize}
\item If forecast errors no longer have zero mean, we are in a setting with interdependent valuations, multi-dimensional, independent types. Indeed, the types of wind producers appear in the valuations of flexible generators but they are independent of the types of flexible generators. On the other hand, if forecast errors are no longer considered independent, correlation coefficients will need to be specified too. We therefore end up in a setting with interdependent valuations and a mix of independent and correlated multi-dimensional types (since correlation coefficients are shared by pairs of wind farms).
\item The choice function may be made belief-dependent; should clarify what this means.
\item In the current setting, flexible generators are not explicitly bidding for reserves (i.e., they are not providing explicit reserve bids like they do for power generation). Would it make sense/be useful at all to introduce reserve bids as well?
\item In the definition of the optimisation problem from which the social choice function is derived, the wind production should be turned into a variable. Indeed, in the current formulation, if the load is smaller than the expected wind production, the problem will be infeasible. It is not entirely clear how it should be bounded, however. It would probably make sense to introduce chance constraints on wind generation bounds. 
\item The bid structure of renewable power plants should be clarified: to what product does a "variance bid" correspond? For instance, in the case of flexible generators, a "generation cost/capacity" bid will correspond to some power generation product 
\item In electricity market applications, we don't have much flexibility when picking the choice/allocation function: in most cases, it will be defined as the argmax of some social welfare maximisation formulation including a variety of operational and physical (e.g., power flow) constraints. This means that the only degree of freedom we have when designing electricity market mechanisms are payment functions. 
\item A theoretical result in Roughgarden's book (Chapter 9, one of the corollaries to Theorem 9.37) states that the VCG payment rule is the only one yielding an incentive-compatible mechanism when the social choice function is social welfare maximisation. I guess that this result was derived for basic social welfare maximisation without any constraints like the ones typically found in electricity market applications. To what extent is that result applicable to electricity markets? My guess is that if it does not work for the unconstrained case, it is highly unlikely to work in the constrained case (since the former can be seen as a relaxation of the latter).
\item What would an application of the VCG mechanism look like for a variance product? See sections above.
\item Most of the basic literature on incentive compatibility in mechanism design makes use of the concept of dominant-strategy incentive-compatibility (DSIC). Would it make sense to relax these requirements and work with different notions of incentive compatibility (e.g., Bayesian-Nash or ex-Post Nash incentive compatibility)? Yes, this has been widely pursued
\item The question of whether the allocation computed via the choice function coincides with a competitive equilibrium does not seem to be directly related to the matter of incentive compatibility for the mechanism at hand: we can thus analyse these issues separately.
\item It's not entirely clear to me what the relation between competitive equilibria and game theoretic equilibria is, however.
\item What is the connection between complementarity/equilibrium models like the one I studied and mechanism design/game theoretic models?
\item interpretation of day-ahead + real-time market as a multi-round auction? implications for 
\end{itemize}

\bibliographystyle{plain} % We choose the "plain" reference style
\bibliography{references_md}

\end{document}