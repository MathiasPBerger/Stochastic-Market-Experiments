\documentclass{article}
\usepackage[utf8]{inputenc}
\usepackage{amsmath}
\usepackage{amsthm}
\usepackage{amssymb}
\usepackage{mathabx}
\usepackage{xcolor}
\newtheorem{theorem}{Theorem}
\newtheorem{proposition}{Proposition}
\newtheorem{definition}{Definition}
\newtheorem{corollary}{Corollary}
\newtheorem{lemma}{Lemma}
\newtheorem{assumption}{Assumption}

\title{Game Theory \& Mechanism Design Primer}
\author{Mathias Berger}
\date{Spring 2022}

\begin{document}

\maketitle

\section{Prelude}

We informally introduce some useful concepts and definitions from the fields of economics, game theory and mechanism design.

\subsection{Economics}

\begin{definition}
(Pareto Optimality) An outcome from which any attempt to benefit a given player by deviating to some other outcome will necessarily result in a loss in utility to someone else.
\end{definition}

In economics, the concept of \textit{efficiency} is typically understood as a Pareto-optimal outcome.

\begin{definition}
(Competitive Equilibrium) \textcolor{red}{TBA}
\end{definition}

\begin{definition}
(Linear Prices) \textcolor{red}{TBA}
\end{definition}

\begin{definition}
(Price-Taking) A situation where producers cannot single-handedly influence market outcomes and prices.
\end{definition}  

\subsection{Game Theory}

In its basic form, a game is as follows.

\begin{definition}
(Game)
\end{definition}

\begin{definition}
(Full-Information Game) A game in which all players know the utilities and strategies of other players.
\end{definition}

\begin{definition}
(Bayesian Game) A game in which players have limited information about the utilities and strategies of other players.
\end{definition}

\begin{definition}
(Non-Cooperative Game) A game in which all players act selfishly and seek to deviate alone from proposed solutions.
\end{definition}

\begin{definition}
(Cooperative Game) A game in which groups of players coordinate their actions.
\end{definition}

\begin{definition}
(Pure Strategy) A complete definition of how a player will play a game.
\end{definition}

\begin{definition}
(Mixed Strategy) An assignment of a probability to each pure strategy and subsequent random selection of the played strategy.
\end{definition}

In other words, a mixed strategy can be interpreted as a randomised strategy, whereby a player picks a probability distribution over her set of pure strategies and independently selects a strategy at random based on this distribution.

\begin{definition}
(One-Shot Simultaneous Move Game) A game in which all players simultaneously choose an action from their set of possible strategies.
\end{definition}

\begin{definition}
(Dominant Strategy) The strategy chosen by a player is dominant if it maximises her utility, irrespective of what other players do.
\end{definition}

\begin{definition}
(Solution Concept) \textcolor{red}{TBA}
\end{definition}

For non-cooperative games, different types of solution concepts can be employed.

\begin{definition}
(Dominant Strategy Equilibrium) An outcome in which every player is doing the best she possibly can irrespective of what other players do.
\end{definition}

\begin{definition}
(Nash Equilibrium) An outcome in which every player is doing the best she possibly can given other players' choices: no player can benefit from unilaterally changing her choice.
\end{definition}

Note that a Nash equilibrium may be strictly Pareto inefficient. This is notably the case in the Prisoner's dilemma, in the sense that there is an outcome other than the (unique) Nash equilibrium in which all of the players achieve a smaller cost.

\begin{definition}
(Pure-Strategy Nash Equilibrium) A Nash equilibrium in which each player deterministically plays her chosen strategy.
\end{definition}

In some settings, pure-strategy Nash equilibria may not exist, which leads to the introduction of the following concept.

\begin{definition}
(Mixed-Strategy Nash Equilibrium) A Nash equilibrium in which each player plays a mixed-strategy.
\end{definition}

Note that players are typically assumed to be risk-neutral in mixed-strategy Nash equilibria.

\begin{definition}
(Bayesian Nash Equilibrium) \textcolor{red}{TBA}
\end{definition}

In the context of cooperative games, the Nash equilibrium concept can be extended as follows.

\begin{definition}
(Strong Nash Equilibrium) An outcome in which no subset of players has a way to simultaneously change their strategies in order to improve each of the participant's welfare.
\end{definition}

\begin{definition}
(Correlated Equilibrium) \textcolor{red}{TBA}
\end{definition}

\subsection{Mechanism Design}

Roughly, mechanism design aims to design games that have dominant strategy solutions and such that these solutions lead to desirable outcomes (e.g., socially desirable). In mechanism design and auction theory, an efficient outcome is sometimes defined as an outcome maximising the sum of utilities of all players (i.e., maximising social welfare).

\begin{definition}
(Direct Mechanism) A direct mechanism is defined by a social choice function (which maps players' preferences or valuations to an outcome or allocation) and a set of payment functions (which determine how players are rewarded or penalised).
\end{definition}

\begin{definition}
(Incentive Compatibility) A mechanism is incentive compatible if it is a dominant strategy for each player to report her private information truthfully.
\end{definition}

\begin{definition}
(Implementation) \textcolor{red}{TBA}
\end{definition}

\begin{definition}
(Shill Bidding) \textcolor{red}{TBA}
\end{definition}

\begin{definition}
(Collusion) \textcolor{red}{TBA}
\end{definition}

\begin{proposition}
(Revelation Principle) If there exists an arbitrary mechanism that implements a given social choice function in dominant strategies, then there exists an incentive compatible mechanism that implements this same social choice function. The payments of the players in the incentive compatible mechanism are identical to those, obtained at equilibrium, of the original mechanism.
\end{proposition}

\begin{proposition}
(Uniqueness of Prices) Given a social choice function and a set of payment functions defining an incentive compatible mechanism, a new mechanism constructed by taking the same social choice function and changing payment functions is incentive compatible if and only if the payment function of each player in the new mechanism can be expressed as the sum of i) the payment function of the same player in the original mechanism, ii) some function that does not depend on the valuation function of this player.
\end{proposition}

\begin{corollary}
(Social Welfare and Incentive Compatibility) The only incentive compatible mechanisms that maximise social welfare are those with Vickrey-Clarke-Groves payments.
\end{corollary}

\begin{definition}
(Incentive Efficiency) \textcolor{red}{TBA}
\end{definition}

\begin{definition}
(Price of Anarchy) The price of anarchy of a game, for some choice of objective function and equilibrium concept, is defined as the ratio between the worst objective function value of an equilibrium of the game and that of an optimal outcome (i.e., an outcome optimising the chosen objective).
\end{definition}

We would ideally like the price of anarchy to be as close to 1 as possible. Note that in the case of games with multiple equilibria, the price of anarchy will be large even if only one of these equilibria is highly inefficient. 

\begin{definition}
(Price of Stability) The price of stability, for some choice of objective function and equilibrium concept, is a measure of inefficiency designed to differentiate between games in which all equilibria are inefficient and those in which some equilibrium is inefficient. It can be computed as the ratio between the best objective function value of one of its equilibria and that of an optimal outcome (i.e., an outcome optimising the chosen objective).
\end{definition}

Note that the price of anarchy and stability are the same in games with a unique equilibrium.

\section{Formulation}

We consider the case of a wholesale electricity market with inelastic demand and model it as a reverse auction. Let us consider a set $\mathcal{G} = \{1, \ldots, n\}$ of power generators and a market operator. We consider a set of possible market outcomes $\mathcal{D} = \bigtimes_g \mathcal{D}_g$, where $\mathcal{D}_g \subseteq \mathbb{R}^{m_g}$ represents the set of possible market outcomes of generator $g$. We assume that each generator $g \in \mathcal{G}$ has cost function $c_g(\hat{\beta}_g): \mathcal{D} \rightarrow \mathbb{R}_+$. More precisely, the cost function $c_g(\beta_g)$ of generator $g$ is parametrised by $\beta_g \in \mathcal{B}_g$, where $\mathcal{B}_g$ defines the set of bids that may be placed by generator $g$ in the market, and the true parameter value $\hat{\beta}_g$ is private. In what follows, we will use $c_g = c_g(\beta_g)$ and $\hat{c}_g = c_g(\hat{\beta}_g)$ as shorthand where appropriate. Note that $\beta_g$ may be a vector or scalar parameter, depending on the context, and it usually has non-negative entries. The bid profile is defined as $\mathcal{B} = \bigtimes_g \mathcal{B}_g$. Then, we define our mechanism as the combination of a social choice function $f: \mathcal{B} \rightarrow \mathcal{D}$ and a set of payment functions $p_g: \mathcal{B} \rightarrow \mathbb{R}, \forall g \in \mathcal{G}$. The social choice function can be interpreted as mapping a bid profile to market outcomes, while payment functions map a bid profile to payments made to each generator. We assume that generators have quasilinear preferences, that is, the utility of each generator is defined as $u_g(\beta) = p_g(\beta) - \hat{c}_g\big(f(\beta)\big)$.

\section{Application to Chance-Constrained Markets}

\section{Comments}

\begin{itemize}
\item In the chance-constrained setting, the cost of flexible generators will depend upon the variance/covariance of the forecast error. If we assume that the latter is only known by wind producers, we would probably need some concept of Bayesian-Nash equilibrium since this implies that flexible generators will need to make assumptions about other players' information to compute their own cost. Is this setting even possible in standard game theory? or do we generally assume that all players can compute their own costs based on their own private information (type)
\end{itemize}

\bibliographystyle{plain} % We choose the "plain" reference style
\bibliography{references}

\end{document}