\documentclass{article}
\usepackage[utf8]{inputenc}
\usepackage{amsmath}
\usepackage{amsthm}
\usepackage{amssymb}
\usepackage{xcolor}
\newtheorem{theorem}{Theorem}
\newtheorem{proposition}{Proposition}
\newtheorem{definition}{Definition}

\title{Properties of Chance-Constrained Stochastic Electricity Markets}
\author{Mathias Berger}
\date{Spring 2022}

\begin{document}

\maketitle

\section{Introduction}

We analyse the properties of a chance-constrained stochastic market for energy and reserves involving stochastic wind power producers, flexible generators and inflexible electricity consumers. 

\section{Problem Statement}

\subsection{Preliminaries}

We consider four types of agents, namely flexible electricity generators, stochastic wind power producers and inflexible electricity consumers, which we describe further below.

\textit{Wind Producers}: We consider a set of stochastic wind farms. Although a forecast $\tilde{W} \in \mathbb{R}_+$ is available for the aggregate production from wind farms, the actual aggregate output $\mathbf{p}_w(\boldsymbol{\omega}) \in \mathbb{R}_+$ may deviate from $\tilde{W}$ by some amount given by random variable $\boldsymbol{\omega} \in \Omega \subseteq \mathbb{R}$, such that $\mathbf{p}_w(\boldsymbol{\omega}) = \tilde{W} + \boldsymbol{\omega}$. Note that we use bold symbols to denote random variables. The first and second-order moments (i.e., the mean and variance) of the distribution of the forecast error $\boldsymbol{\omega}$ are denoted by $\mathbb{E}[\boldsymbol{\omega}] = \mu$ and $\mbox{Var}[\boldsymbol{\omega}] = \sigma^2$, respectively. All agents are assumed to have access to the same amount of information about $\mathbf{p}(\boldsymbol{\omega})$. Wind producers are assumed to have zero marginal cost.

\textit{Flexible Generators}: We consider $K \in \mathbb{N}$ dispatchable generators. Each generator may produce electricity and contribute to the provision of reserves in the system. Hence, the actual power output $\mathbf{p}_k(\boldsymbol{\omega})$ is given by an affine control law. For generator $k$, this law can be expressed as $\mathbf{p}_k(\boldsymbol{\omega}) = p_k - \alpha_k \boldsymbol{\omega}$, where $p_k$ denotes the scheduled power output and $\alpha_k$ is the share of the forecast error covered by generator $k$ through the reserve mechanism. Since the actual power output of generators directly depends on the uncertain wind production and only becomes known when $\boldsymbol{\omega}$ is revealed, it must be ensured that power generation bounds are not exceeded. Chance (i.e., probabilistic) constraints can be used for this purpose, and $\epsilon_k$ denotes the tolerance for constraint violations (e.g., constraints may be violated $100 \times \epsilon_k \%$ of the time). Generator $k$ is assumed to have linear and quadratic marginal production cost components denoted by $c_k^L \in \mathbb{R}_+$ and $c_k^Q \in \mathbb{R}_+$, respectively.

\textit{Inflexible Consumers}: We consider a set of consumers with aggregate demand $D \in \mathbb{R}_+$, which is assumed inelastic and known with certainty.

\textit{Market Operator}: A market operator (also known as a \textit{social planner}) seeks to identify prices for energy and reserves so that the markets for energy and reserves clear. \textcolor{orange}{clarify its role in the equilibrium problem formulation; allocate energy production and reserve procurement across generators such that 1) the market clears and 2) the prices and decisions maximise the profits of generators}. 
{\color{blue}[rom: I think the equilibrium formulation is more of a tool to analyse optimal behavior in of the agents in the competitive setting. Maximizing the profits of the gens is flawed for inelastic demand. Ideally, we want welfare maximization (cost minimization in case of inelastic demand), without addtional constraints on cost recovery. (Also note that we are ignorning any long-term proftiability here.)]}


\section{Equilibrium Problem Formulation}

The stochastic electricity market can be modelled as a stochastic equilibrium problem where flexible generators seek to maximise their utility while being coupled through market clearing constraints. The different components of the equilibrium problem formulation are described in this section. 

\subsection{Flexible Generators}

We start by describing the stochastic optimisation problems faced by flexible generators, and recast them as deterministic optimisation problems.

\subsubsection{Stochastic Formulation}

We consider risk-neutral flexible generators (i.e., \textcolor{orange}{generators only care about the expected pay-off and not its distribution around the mean}). Thus, flexible generator $k$ seeks to maximise its expected profit subject to chance constraints on its power output:

\begin{align}
\underset{p_k, \alpha_k}{\max} \hspace{10pt} & \mathbb{E}[\lambda \mathbf{p}_k(\boldsymbol{\omega}) + \chi \alpha_k - c_k^L \mathbf{p}_k(\boldsymbol{\omega}) - c_k^Q (\mathbf{p}_k(\boldsymbol{\omega}))^2]\\
\mbox{s.t. } & \mathbb{P}[\mathbf{p}_k(\boldsymbol{\omega}) \le \overline{p}_k] \ge 1 - \epsilon_k,\\
& \mathbb{P}[0 \le \mathbf{p}_k(\boldsymbol{\omega})] \ge 1 - \epsilon_k,\\
&p_k \in \mathbb{R}, 0 \le \alpha_k \le 1.
\end{align}
Note that electricity and reserve prices, which are denoted by $\lambda$ and $\chi$, respectively, are treated as parameters by each producer $k$.
\subsubsection{Deterministic Equivalent Formulation}

The stochastic expected profit maximisation problem faced by producer $k$ can be recast as an equivalent deterministic program as follows. 

First, by linearity of the expectation operator, the expected profit can be successively re-written as
\begin{align*}
\Pi_k(p_k, \alpha_k) &= \mathbb{E}[\lambda \mathbf{p}_k(\boldsymbol{\omega}) + \chi \alpha_k - c_k^L \mathbf{p}_k(\boldsymbol{\omega}) - c_k^Q (\mathbf{p}_k(\boldsymbol{\omega}))^2]\\
&= \mathbb{E}[\lambda \mathbf{p}_k(\boldsymbol{\omega})] + \mathbb{E}[\chi \alpha_k] - \mathbb{E}[c_k^L \mathbf{p}_k(\boldsymbol{\omega})] - \mathbb{E}[c_k^Q (\mathbf{p}_k(\boldsymbol{\omega}))^2].
\end{align*}
We focus on each term separately and obtain
\begin{align*}
\mathbb{E}[\lambda \mathbf{p}_k(\boldsymbol{\omega})] &= \mathbb{E}[\lambda (p_k - \alpha_k \boldsymbol{\omega})]\\
& = \lambda (p_k - \alpha_k \mu),\\
\mathbb{E}[\chi \alpha_k] &= \chi \alpha_k,\\
\mathbb{E}[c_k^L \mathbf{p}_k(\boldsymbol{\omega})] &= \mathbb{E}[c_k (p_k - \alpha_k \boldsymbol{\omega})]\\
&= c_k^L(p_k - \alpha_k \mu),\\
\mathbb{E}[c_k^Q (\mathbf{p}_k(\boldsymbol{\omega}))^2] &= \mathbb{E}[c_k^Q (p_k - \alpha_k \boldsymbol{\omega})^2]\\
&= \mathbb{E}[c_k^Q (p_k^2 - 2 \boldsymbol{\omega} \alpha_k p_k + \alpha_k^2 \boldsymbol{\omega}^2)]\\
&= c_k^Q (p_k^2 - 2\mu \alpha_k p_k + \alpha_k^2 \mathbb{E}[\boldsymbol{\omega}^2])\\
&= c_k^Q (p_k^2 - 2\mu \alpha_k p_k + \alpha_k^2 (\mbox{Var}[\boldsymbol{\omega}] + \mathbb{E}[\boldsymbol{\omega}]^2))\\
&= c_k^Q (p_k^2 - 2\mu \alpha_k p_k + \alpha_k^2 (\sigma^2 + \mu^2))\\
&= c_k^Q \big((p_k - \alpha_k \mu)^2 + \alpha_k^2 \sigma^2\big).
\end{align*}
Thus, the expected profit of producer $k$ becomes
\begin{equation*}
\Pi_k(p_k, \alpha_k) = \lambda (p_k - \alpha_k \mu) + \chi \alpha_k - c_k^L(p_k - \alpha_k \mu) - c_k^Q \big((p_k - \alpha_k \mu)^2 + \alpha_k^2 \sigma^2\big).
\end{equation*}
The first term represents the expected revenue accruing from the sale of electricity. Note that this revenue is computed based on the expected production level and factors in reserve procurement decisions. The second term is the expected revenue from reserve services. The third term represents the expected linear cost component. The last term represents the expected quadratic cost component. Note that the quadratic cost component depends both on the expected production level and the variance of the forecast error.

Then, the chance constraints imposed on generation bounds should also be re-formulated to obtain a tractable (hopefully convex) mathematical program. Depending on the underlying forecast error distribution, chance constraints may be recast as convex constraints involving moments of said distribution. In particular, one-sided chance constraints involving a scalar random variable following a Gaussian distribution or approximations of distributionally-robust chance constraints may lead to simple linear inequality constraints \cite{Dvorkin2020}. In other settings, distributionally-robust chance constraints may be recast as second-order cone constraints \cite{Xie2018}. For simplicity, we assume that $\boldsymbol{\omega} \sim \mathcal{N}(\mu, \sigma^2)$. The derivation then goes as follows:
\begin{align*}
&\mathbb{P}[\mathbf{p}_k(\boldsymbol{\omega}) \le \overline{p}_k] \ge 1 - \epsilon_k\\
\Leftrightarrow &\mathbb{P}[p_k - \alpha_k \boldsymbol{\omega} \le \overline{p}_k] \ge 1 - \epsilon_k\\
\Leftrightarrow &\mathbb{P}\Big[\frac{- \alpha_k \boldsymbol{\omega} + \alpha_k \mu}{\sqrt{\alpha_k^2 \sigma^2}} \le \frac{\overline{p}_k - p_k + \alpha_k \mu}{\sqrt{\alpha_k^2 \sigma^2}}\Big] \ge 1 - \epsilon_k\\
\Leftrightarrow &\frac{\overline{p}_k - p_k + \alpha_k \mu}{\alpha_k \sigma} \ge \Phi^{-1}(1 - \epsilon_k)\\
\Leftrightarrow &p_k \le \overline{p}_k - \alpha_k \big(\Phi^{-1}(1 - \epsilon_k)\sigma - \mu\big)\\
\Leftrightarrow &p_k \le \overline{p}_k - \alpha_k \phi_k,
\end{align*}
where $\Phi^{-1}: (0, 1) \rightarrow \mathbb{R}$ is the quantile function of the standard normal distribution and we define $\phi_k = \Phi^{-1}(1 - \epsilon_k)\sigma - \mu$. Applying the same ideas to the second chance constraint yields
\begin{align*}
&\mathbb{P}[0 \le \mathbf{p}_k(\boldsymbol{\omega})] \ge 1 - \epsilon_k\\
\Leftrightarrow & \alpha_k \phi_k \le p_k.
\end{align*}
Thus, the deterministic equivalent to the stochastic profit-maximisation problem of producer $k$ reads
\begin{align}
\underset{p_k, \alpha_k}{\max} \hspace{10pt} & \lambda (p_k - \alpha_k \mu) + \chi \alpha_k - c_k^L(p_k - \alpha_k \mu) - c_k^Q \big((p_k - \alpha_k \mu)^2 + \alpha_k^2 \sigma^2\big)\\
\mbox{s.t. } & p_k \le \overline{p}_k - \alpha_k \phi_k, \hspace{25pt} (\overline{\nu}_k)\\
& \alpha_k \phi_k \le p_k, \hspace{48pt}(\underline{\nu}_k)\\
&p_k \in \mathbb{R}, 0 \le \alpha_k \le 1,
\end{align}
which is a (convex) quadratic program.
\subsection{Market Operator}

\subsubsection{Stochastic Formulation}

The market operator seeks to clear the energy market and procure enough reserves to cover any forecast error,
\begin{align}
& \sum_k \mathbf{p}_k(\boldsymbol{\omega}) + \mathbf{p}_w(\boldsymbol{\omega}) = D,\\
& \sum_k \alpha_k = 1.
\end{align}

\subsubsection{Deterministic Equivalent Formulation}

Simply substituting the affine control law of flexible generators yields
\begin{align}
& \sum_k p_k + \tilde{W} = D, \hspace{10pt} (\lambda)\\
& \sum_k \alpha_k = 1, \hspace{37pt} (\chi)
\end{align}
\textcolor{orange}{where electricity and reserve prices are obtained as the dual variables of the power balance and reserve allocation constraints.}
{\color{blue}[rom: While I do think using $\chi$ is the most promising avenue, there are of course other ways to do this, for example directly pricing variance (or standard deviaton) and obtaining the price as the marginal cost of a unit of variance \cite{brooks2022locational}]}


\subsection{Wind Producers}

Wind producers are price takers and do not control their output (i.e., no spillage is allowed), and therefore do not solve any optimisation problem \textit{per se}. Their expected revenue can be computed as $R = \mathbb{E}[\lambda \mathbf{p}_w(\boldsymbol{\omega})] = \lambda(\tilde{W} + \mu)$.

\subsection{Electricity Consumers}

The demand is assumed to be inelastic and electricity consumers do not solve any optimisation problem either. Their expected pay-off can be calculated as $P = \mathbb{E}[\lambda D] = \lambda D$.


\section{Market Clearing Problem Formulation}

This section describes a stochastic market clearing problem that can be solved by a market operator in order to identify prices for energy and reserves, schedule generators and allocate reserves in a cost-efficient way. 

\subsection{Stochastic Formulation}
Under the assumption that the market operator is risk-neutral, the stochastic market clearing problem can be formulated as

\begin{align}
\underset{\{p_k, \alpha_k\}_{\forall k}}{\min} \hspace{10pt} & \mathbb{E}\Big[\sum_k \big(c_k^L \mathbf{p}_k(\boldsymbol{\omega}) + c_k^Q (\mathbf{p}_k(\boldsymbol{\omega}))^2\big)\Big]\\
\mbox{s.t. } & \mathbb{P}[\mathbf{p}_k(\boldsymbol{\omega}) \le \overline{p}_k] \ge 1 - \epsilon_k, \mbox{ }\forall k,\\
& \mathbb{P}[0 \le \mathbf{p}_k(\boldsymbol{\omega})] \ge 1 - \epsilon_k, \mbox{ }\forall k,\\
& \sum_k \mathbf{p}_k(\boldsymbol{\omega}) + \mathbf{p}_w(\boldsymbol{\omega}) = D,\\
& \sum_k \alpha_k = 1, \\
& p_k \in \mathbb{R}, 0 \le \alpha_k \le 1.
\end{align}

\subsection{Deterministic Equivalent Formulation}
Using techniques introduced earlier yields the following equivalent deterministic formulation of the market clearing problem:
\begin{align}
\underset{\{p_k, \alpha_k\}_{\forall k}}{\min} \hspace{10pt} & \sum_k \Big(c_k^L(p_k - \alpha_k \mu) + c_k^Q \big((p_k - \alpha_k \mu)^2 + \alpha_k^2 \sigma^2\big)\Big)\\
\mbox{s.t. } & p_k \le \overline{p}_k - \alpha_k \phi_k, \mbox{ }\forall k, \hspace{15pt}(\underline{\nu}_k)\\
& \alpha_k \phi_k \le p_k, \mbox{ }\forall k, \hspace{37pt}(\underline{\nu}_k)\\
& \sum_k p_k + \tilde{W} = D, \hspace{30pt} (\lambda)\\
& \sum_k \alpha_k = 1,\hspace{55pt} (\chi) \\
& p_k \in \mathbb{R}, 0 \le \alpha_k \le 1,
\end{align}
where the dual variables of the energy balance and reserve allocation constraints are used to set energy and reserve prices, respectively. Note that computing expectations and treating chance constraints in the problem solved by the market operator in the same way as in the problem of a market participant is only possible under the assumption that they all share the same information about $\boldsymbol{\omega}$. It will become apparent later that this is also a pre-requisite for the pricing scheme of the market operator to support a competitive equilibrium (since the complementarity problems stemming from the equilibrium and market clearing approaches would otherwise be different). This topic is also discussed in the context of a stochastic programming-based market design by Dvorkin et al. \cite{DvorkinV2019}.
\section{Market Properties}

\textcolor{red}{Propositions 1 and 2 should be updated}.

We analyse four key properties of the proposed stochastic market design, namely whether it supports a \textit{competitive equilibrium}, guarantees \textit{cost recovery} for flexible generators, \textit{revenue adequacy} for the market operator, and is \textit{incentive compatible}.

\begin{definition}
(Competitive Equilibrium) A competitive equilibrium for the stochastic market is a set prices $\{\lambda^*, \chi^*\}$ and decisons $\{p_k^*, \alpha_k^*\}_{\forall k}$ that\vspace{-5pt}
\begin{enumerate}
\item clear the market: $\sum_k p_k^* + \tilde{W} = D$ and $\sum_k \alpha_k^* = 1$\vspace{-5pt}
\item maximise the profit of flexible generators
\end{enumerate}
\end{definition}

\begin{proposition}
(\textcolor{green}{Competitive Equilibrium}) The stochastic market clearing problem produces prices and decisions maximising the profit of each flexible generator and supporting a competitive equilibrium from which they have no incentive to deviate.
\end{proposition}
\begin{proof}
The proof showing that decisions $\{p_k^*, z_k^*, \alpha_k^*\}$ and prices $\{\lambda^*, \chi^*, \mu_k^*\}$ support a competitive equilibrium proceeds in two steps. The first step consists in finding the (negative) profit earned by each producer for prices and decisions computed by the market operator. The second step consists in showing that this profit is optimal for the problem faced by each producer.

\textit{First Step:} Let $V_k^*$ denote the value of the objective of producer $k$ under said market prices and decisions
\begin{align}
    V_k^* =& c_k p_k^* + f_k z_k^* - \lambda^* p_k^* - \chi^* \alpha_k^* - \mu_k^* z_k^* \\
    =& c_k p_k^* + f_k z_k^* - \lambda^* p_k^* - \chi^* \alpha_k^* - \mu_k^* z_k^* + \underline{\mu}_k^*(-p_k^*  + \underline{p}_k z_k^* + \phi_{\epsilon} \sigma \alpha_k^*)\\
    & + \overline{\mu}_k^* (-\overline{p}_k z_k^* + \phi_{\epsilon} \sigma \alpha_k^* + p_k^*) + v_k^*(\alpha_k^* - z_k^*)\\
    =& (c_k - \lambda^* - \underline{\mu}_k^* + \overline{\mu}_k^*) p_k^*+ (f_k - \mu_k^* + \underline{\mu}_k^* \underline{p}_k - \overline{\mu}_k^* \overline{p}_k - v_k^*) z_k^*\\
    &+ (-\chi^* + \phi_{\epsilon} \sigma \underline{\mu}_k^* + \phi_{\epsilon} \sigma \overline{\mu}_k^* + v_k^*) \alpha_k^*\\
    =& 0
\end{align}
The second line results from the fact that the three new terms added to the objective are equal to zero (complementary slackness). The third line is obtained by re-arranging terms. The fourth line follows from the fact that the first two terms in parentheses are equal to zero (dual feasibility of market prices) and the last term is also equal to zero (complementary slackness).

{\color{blue}[This is a bit hard to follow. For example I am missing some explanation of $f_k$ and $z_k$. Also $\overline{\mu}$ and $\underline{\mu}$ Can you repeat the exact version of the producers problem that you are using for this derivation somewhere? It's a bit weird to me that you are getting a zero profit for all producers, but I would have to go thorugh that more thoroughly.]}

\textit{Second Step:} Now, let $\{p_k^\star, z_k^\star, \alpha_k^\star\}$ denote the solution of the problem faced by producer $k$ under prices $\{\lambda^*, \chi^*, \mu_k^*\}$, and let $V_k^\star$ denote the corresponding objective. One successively finds
\begin{align}
    V_k^\star =& c_k p_k^\star + f_k z_k^\star - \lambda^* p_k^\star - \chi^* \alpha_k^\star - \mu_k^*z_k^\star\\
    \ge& (c_k p_k^\star + f_k z_k^\star - \lambda^* p_k^\star - \chi^* \alpha_k^\star - \mu_k^*z_k^\star) + v_k^*(\alpha_k^\star - z_k^\star)\\
    &+ \underline{\mu}_k^*(-p_k^\star  + \underline{p}_k z_k^\star + \phi_{\epsilon} \sigma \alpha_k^\star) + \overline{\mu}_k^* (-\overline{p}_k z_k^\star + \phi_{\epsilon} \sigma \alpha_k^\star + p_k^\star)\\
    =& (c_k - \lambda^* - \underline{\mu}_k^* + \overline{\mu}_k^*) p_k^\star+(f_k - \mu_k^* + \underline{\mu}_k^* \underline{p}_k - \overline{\mu}_k^* \overline{p}_k - v_k^*) z_k^\star\\
    &+ (-\chi^* + \phi_{\epsilon} \sigma \underline{\mu}_k^* + \phi_{\epsilon} \sigma \overline{\mu}_k^* + v_k^*) \alpha_k^\star\\
    \ge& 0\\
    =& V_k^*
\end{align}
The second line stems from the fact that market prices are dual feasible (thus nonnegative), while terms in parentheses are nonpositive (primal feasibility for problem faced by producer $k$), such that the product terms added to the objective are nonpositive. The third line is obtained by re-arranging terms. The fourth line follows from the fact that 1) the first two terms in parentheses are equal to zero (dual feasibility of market prices), 2) the third term in parentheses is nonnegative (dual feasibility of market prices) and $\alpha_k^\star \ge 0$ (primal feasibility in profit-maximisation problem of producer $k$). Thus, $V_k^\star \ge V_k^*, \forall k$. In other words, market prices and decisions maximise the profit of each producer, resulting in a competitive equilibrium.
\end{proof}

\begin{proposition}
(\textcolor{orange}{Cost Recovery}) The stochastic market clearing problem produces prices and decisions that guarantee a nonnegative pay-off for flexible generators.
\end{proposition}
\begin{proof}
Can be shown by taking the dual problem of profit-maximising agent where the commitment decision is enforced and using strong duality (\textcolor{red}{but strong duality doesn't hold for these problems, however. In Yury's paper, this is circumvented by assuming that the commitment decisions are already provided but I am not sure how that can be justified. This would also imply that profits $\ge 0$, while the competitive equilibrium proof suggests that profits $= 0$. How do we make sense of this?}).
{\color{blue} [rom: The commitment decisions are not ''provided'' in the sense that they are external to the problem. Rather, it can be shown that when you solve a MIP and then use the resulting decisons on integer variables as parameters and run the problem again as an LP (with the previous integer variables now fixed as a parameter), you get the same result as the oringinal MIP and can apply convexity/strong duality. This concept was made popular for the power community here \cite{o2005efficient}. However, it is a bit dangerous to apply without further thought, because the dual on the fixed integer variable can be a price proxy for commitment but will for sure be off because it can not capture the true cost from the non-continous commitment decision. (That's why my papers usually do not discuss unit commitment but assume a pre-committed dispatch as one might see in a real-time market.)]}

\end{proof}

\begin{proposition}
(\textcolor{red}{Revenue Adequacy}) The stochastic market clearing problem \textcolor{red}{does not} produce prices and decisions guaranteeing that the market operator does not incur any financial loss in expectation.
\end{proposition}
\begin{proof}
Let $\{\lambda^*, \chi^*\}$ and $\{\{p_k^*\}_{\forall k}, \{\alpha_k^*\}_{\forall k}\}$ denote prices and decisions calculated by the market operator. Consumers only pay the market operator for the electricity consumed, while producers receive payments covering energy and reserves. Hence, the surplus collected by the market operator can be expressed as follows
\begin{align*}
\boldsymbol{\Delta}^* &= \lambda^*D - \lambda^*\mathbf{p}_w(\boldsymbol{\omega}) - \sum_k \lambda^*\mathbf{p}_k(\boldsymbol{\omega})\\
&= \lambda^*D - \lambda^*(\tilde{W} + \boldsymbol{\omega}) - \sum_k \lambda^*(p_k^* - \alpha_k^* \boldsymbol{\omega}) - \sum_k \alpha_k^* \chi^*\\
&= \lambda^*\big(D - \tilde{W} - \boldsymbol{\omega} - \sum_k (p_k^* - \alpha_k^* \boldsymbol{\omega})\big) - \chi^* \sum_k \alpha_k^*\\
&= -\chi^*
\end{align*}
%\begin{align*}
%\mathbb{E}[\Delta^*] &= \mathbb{E}\Big[\lambda^*D - \lambda^*(\tilde{W} + \boldsymbol{\omega}) - \sum_k \lambda^*(p_k^* - \alpha_k^* \boldsymbol{\omega}) - \sum_k \alpha_k^* \chi^*\Big]\\
%&= \mathbb{E}\Big[\lambda^*\big(D - \tilde{W} - \boldsymbol{\omega} - \sum_k (p_k^* - \alpha_k^* \boldsymbol{\omega})\big) - \chi^* \sum_k \alpha_k^*\Big]\\
%&= \mathbb{E}[-\chi^*]\\
%&= -\chi^*
%\end{align*}
where the fourth equality stems from the fact that the market clears (power balance and reserve procurement constraints are enforced in the primal problem solved by the market operator) for any realisation of $\boldsymbol{\omega}$. Since $\chi^* \in \mathbb{R}$, \textcolor{red}{it may happen that $\boldsymbol{\Delta}^* < 0$}, which implies that the market operator may incur a financial deficit.

{\color{blue}[This is correct, I have a proof lying around somewhere that the cost can be recovered exactly by reducing the $\lambda$ paid to wind power producers by a value that is proportional to their forecast uncertainty $\sigma$. Unfortunately, this falls apart when network is considered. This is one of the things that I mentioned that I wanted to clear up for a while but never got to it yet.]}

\end{proof}

\begin{proposition}
(\textcolor{red}{Incentive Compatibility}) The stochastic market clearing problem \textcolor{red}{does not produce} prices and decisions guaranteeing that each flexible generator maximises its profit by bidding truthfully.
\end{proposition}
\begin{proof}
See counter-example implemented in Julia \cite{SMER2022}, where a producer operating in an electricity market with a finite number of producers can increase its profit by bidding untruthfully.
\end{proof}

Market properties are also discussed in \cite{Ratha2019} in a similar set-up. The proof of incentive compatibility seems wrong, however, and there are also errors in the proof of the revenue adequacy property. 
{\color{blue}[I remember reading this paper and thinking that it felt overengineered, would like to discuss what you think the flaws are.]}

\section{Comments}

The profit accrued to producer $k$ is nonpositive (or equal to 0). How should this be interpreted?
{\color{blue}[As mentioned above I am a bit confused on why you get nonpostiive profits for all producers. The interpreteation of low or negative profits for most of the producers, however, is I beliefe a standard result of marginal cost-based pricing. It should be assumed, that the "maringal cost", i.e. the cost functions of the producers bid into the market account for long-term profitability somehow.]}

\section{Appendix A}
This section derives complementary problems based on the KKT conditions of the deterministic equivalent of the equilibrium and market problems and \textcolor{red}{shows that they are identical (hence the equivalence between solutions of the two problems}.

\subsection{Equilibrium Problem}
To ease the writing of KKT conditions, the problem faced by producer $k$ is first re-written as a minimisation problem,
%\begin{equation*}
%\lambda (p_k - \alpha_k \mu) + \chi \alpha_k - c_k^L(p_k - \alpha_k \mu) - c_k^Q \big((p_k - \alpha_k \mu)^2 + \alpha_k^2 \sigma^2\big)
%\end{equation*}
\begin{align}
-\underset{p_k, \alpha_k}{\min} \hspace{10pt} & c_k^L(p_k - \alpha_k \mu) + c_k^Q \big((p_k - \alpha_k \mu)^2 + \alpha_k^2 \sigma^2\big) - \lambda (p_k - \alpha_k \mu) - \chi \alpha_k\\
\mbox{s.t. } & p_k \le \overline{p}_k - \alpha_k \phi_k, \hspace{25pt} (\overline{\nu}_k)\\
& \alpha_k \phi_k \le p_k, \hspace{48pt}(\underline{\nu}_k)\\
&p_k \in \mathbb{R}, 0 \le \alpha_k \le 1.
\end{align}
Then, the Lagrangian function of producer $k$ can be expressed as
\begin{align*}
\mathcal{L}_k(p_k, \alpha_k, \lambda, \chi, \underline{\nu}_k, \overline{\nu}_k) =& c_k^L(p_k - \alpha_k \mu) + c_k^Q \big((p_k - \alpha_k \mu)^2 + \alpha_k^2 \sigma^2\big)\\
&- \lambda (p_k - \alpha_k \mu) - \chi \alpha_k + \underline{\nu}_k(\alpha_k \phi_k - p_k)\\
&+ \overline{\nu}_k (p_k - \overline{p}_k + \alpha_k \phi_k).
\end{align*}
The KKT conditions of the problem faced by producer $k$ now follow
\begin{align*}
&\frac{\partial \mathcal{L}_k}{\partial p_k} = c_k^L + 2 c_k^Q (p_k - \alpha_k \mu) - \lambda - \underline{\nu}_k + \overline{\nu}_k = 0\\
&\frac{\partial \mathcal{L}_k}{\partial \alpha_k} = - c_k^L \mu + 2 c_k^Q(\alpha_k(\mu^2 + \sigma^2) - p_k \mu) \textcolor{red}{+ \lambda \mu} - \chi + \underline{\nu}_k \phi_k + \overline{\nu}_k \phi_k = 0\\
&0 \le \underline{\nu}_k \perp p_k - \alpha_k \phi_k \ge 0\\
&0 \le \overline{\nu}_k \perp \overline{p}_k - \alpha_k \phi_k - p_k \ge 0,
\end{align*}

{\color{blue}[rom: I think this can be fixed by interprting $\alpha\mu$ as a scheduled energy component that is reimbursed with $\lambda$]. (So basically a zero-mean assumption with detour).}

where the range constraints for $\alpha_k$ were omitted for conciseness. Gathering the KKT conditions of each producer and adding the market clearing constraints yields the complementarity problem associated with the equilibrium problem.

\subsection{Stochastic Market Clearing Problem}
\textcolor{red}{The KKT conditions do not seem to coincide with the ones of the equilibrum problem (we're missing a term in the $\alpha_k$ stationarity condition): do producers get paid for the scheduled production $p_k$ or the real-time production $\mathbf{p}_k(\boldsymbol{\omega})$? The KKT conditions would be identical if the former held true (the problematic term in the $\alpha_k$ stationarity condition of producer $k$, which is highlighted in red, would then disappear)}
For simplicity, let 
\begin{align*}
x &= \begin{pmatrix} \{p_k\}_{\forall k}, \{\alpha_k\}_{\forall k} \end{pmatrix}^\top,\\
\pi &= \begin{pmatrix} \lambda, \chi, \{\underline{\nu}_k\}, \{\overline{\nu}_k\} \end{pmatrix}^\top,
\end{align*}
denote the set of primal and dual variables, respectively. Let us form the Lagrangian function of the stochastic market clearing problem:
\begin{align*}
\mathcal{L}(x, \pi) =& \sum_k \Big(c_k^L(p_k - \alpha_k \mu) + c_k^Q \big((p_k - \alpha_k \mu)^2 + \alpha_k^2 \sigma^2\big)\Big) + \sum_k \underline{\nu}_k (\alpha_k \phi_k - p_k)\\
& + \sum_k \overline{\nu}_k (p_k - \overline{p}_k + \alpha_k \phi_k) + \lambda(D - \sum_k p_k - \tilde{W}) + \chi(1 - \sum_k \alpha_k)
\end{align*}
The KKT conditions associated with the stochastic market clearing problem therefore read
\begin{align*}
&\frac{\partial \mathcal{L}}{\partial p_k} = c_k^L + 2 c_k^Q (p_k - \alpha_k \mu) - \underline{\nu}_k + \overline{\nu}_k - \lambda = 0, \forall k\\
&\frac{\partial \mathcal{L}}{\partial \alpha_k} = - c_k^L \mu + 2 c_k^Q(\alpha_k(\mu^2 + \sigma^2) - p_k \mu) + \underline{\nu}_k \phi_k + \overline{\nu}_k \phi_k  - \chi = 0, \forall k\\
&\sum_k p_k + \tilde{W} = D\\
&\sum_k \alpha_k = 1\\
&0 \le \underline{\nu}_k \perp p_k - \alpha_k \phi_k \ge 0, \forall k\\
&0 \le \overline{\nu}_k \perp \overline{p}_k -  \alpha_k \phi_k- p_k \ge 0, \forall k
\end{align*}
from which the range constraints of $\alpha_k$ were omitted for conciseness. Since the objective function is convex, the functions forming the constraints are continuously differentiable and the problem satisfies Slater's condition, the KKT conditions are necessary and sufficient. Thus, a solution to the complementarity problem gives a (globally) optimal solution of the market clearing problem. In addition, since the objective is strongly convex and the feasible set is convex (and non-empty), there is also a unique optimal solution to the problem.

\bibliographystyle{plain} % We choose the "plain" reference style
\bibliography{references}

\end{document}